%!TEX root = ../manual.tex

\section{Installation et utilisation}
    \subsection{Installation}
        Copier le répertoire \verb|texlab| n'importe où sur votre disque (la racine de votre project latex peut être une bonne idée). Dans ce manuel, ce répertoire sera copié dans \verb|C:|, et ses fichiers sont donc accessibles dans \verb|C:\texlab|.

        Pour utiliser toutes les fonctionnalités de \verb|texlab|, \href{https://www.python.org/}{python} doit être installé sur votre système, ainsi que le package \verb|pygments|, que l'on peut installer avec la commande
        \begin{code}{shell}
             pip install pygments
        \end{code} 

    \subsection{Mise en place}
        Pour utiliser \verb|texlab|, il suffit de créer un document et de commencer son préambule comme suit :
        \begin{code}{latex}
            \newcommand{\templatesroot}{C:/texlab}
            \documentclass{article}

% \usepackage{amsmath}
% \usepackage{amssymb}

\usepackage{dsfont}
\usepackage{bbold}
\usepackage{fontspec}
\usepackage{etex}

\usepackage[left=2cm,right=2cm,top=2cm,bottom=2cm]{geometry}

\usepackage{hyperref}
\usepackage[nameinlink]{cleveref}

\usepackage{tikz}

\usepackage[many]{tcolorbox}

\usepackage{xcolor}

\usepackage{multicol}
\setlength{\columnseprule}{0.4pt}

\usepackage[french]{babel}

\usepackage{xifthen}

\usepackage{tikz}

% Bullets au lieu de tirets dans les itemize
\renewcommand{\labelitemi}{\bullet}
\usepackage{fontspec}
\usepackage{etex}

\usepackage[left=2cm,right=2cm,top=2cm,bottom=2cm]{geometry}

\usepackage{hyperref}
\usepackage[nameinlink]{cleveref}

\usepackage{tikz}

\usepackage[many]{tcolorbox}

\usepackage{xcolor}

\usepackage{multicol}
\setlength{\columnseprule}{0.4pt}

\usepackage[french]{babel}

\usepackage{xifthen}

\usepackage{tikz}

% Bullets au lieu de tirets dans les itemize
\renewcommand{\labelitemi}{\bullet}
% \usepackage[amsmath,framed,thmmarks,hyperref]{ntheorem}
\usepackage{amsthm}

% Couleurs
\definecolor{theoremcolor}{RGB}{255,117,117}
% \definecolor{definitioncolor}{RGB}{111,214,145}
\definecolor{definitioncolor}{RGB}{46,184,46}
\definecolor{commandcolor}{RGB}{66,173,244}
\definecolor{proofcolor}{RGB}{0,0,0}
\definecolor{exercisecolor}{RGB}{255,199,122}
% \definecolor{exercisecolor}{RGB}{46,145,184}
\definecolor{remarkcolor}{RGB}{137,137,137}
\definecolor{methodcolor}{RGB}{105,188,155}
\definecolor{commentcolor}{RGB}{244,113,66}
\definecolor{subjectcolor}{RGB}{244,116,65}
\definecolor{testexercisecolor}{RGB}{200,200,200}
% \definecolor{todolv1color}{RGB}{0,204,0}
% \definecolor{todolv2color}{RGB}{255,204,0}
% \definecolor{todolv3color}{RGB}{255,153,153}
% \definecolor{bigtodocolor}{RGB}{100,100,100}
\definecolor{todolv1color}{RGB}{111,214,145}
\definecolor{todolv2color}{RGB}{255,199,122}
\definecolor{todolv3color}{RGB}{255,117,117}
\definecolor{bigtodocolor}{RGB}{100,100,100}

\newcounter{dummy}

% Général
\tcbset{
    globalbox/.style={
            enhanced jigsaw,
            boxrule=0pt,
            arc=0mm,
            outer arc=0mm,
            leftrule=2pt,
            bottomrule=0.5pt,
            toprule=0pt,
            rightrule=0.5pt,
            fonttitle=\bfseries\scshape,
            breakable=true,
            separator sign dash,
        }
}

% Théorèmes
\tcbset{
    theorembox/.style={
            colframe=theoremcolor!80,
            colback=theoremcolor!2,
            coltitle=theoremcolor!60!black,
            title style={theoremcolor!30},
        %     fonttitle=\sc
        }
}

\newtcbtheorem[use counter=dummy, number within=section]{theorem}{Théorème}{globalbox,theorembox,label type=theorem}{theorems}
\crefname{theorem}{Théorème}{Théorèmes}

\newtcbtheorem[use counter=dummy, number within=section]{property}{Propriété}{globalbox,theorembox,label type=property}{properties}
\crefname{property}{Propriété}{Propriétés}

\newtcbtheorem[use counter=dummy, number within=section]{corollary}{Corollaire}{globalbox,theorembox,label type=corollary}{corollaries}
\crefname{corollary}{Corollaire}{Corollaires}

\newtcbtheorem[use counter=dummy, number within=section]{lemma}{Lemme}{globalbox,theorembox,label type=lemma}{lemmas}
\crefname{lemma}{Lemme}{Lemmes}

% Définition
\tcbset{
    definitionbox/.style={
            colframe=definitioncolor!70,
            colback=definitioncolor!2,
            coltitle=definitioncolor!60!black,
            title style={definitioncolor!20},
        %     fonttitle=\sc
        }
}

\newtcbtheorem[use counter=dummy, number within=section]{definition}{Définition}{globalbox,definitionbox,label type=definition}{definitions}
\crefname{definition}{Définition}{Définitions}

% Commandes
\tcbset{
    commandbox/.style={
            colframe=commandcolor,
            colback=commandcolor!5,
            coltitle=commandcolor!20!black,
            title style={commandcolor!50},
        %     fonttitle=\sc
        }
}

\newtcbtheorem[use counter=dummy, number within=section]{command}{Commande}{globalbox,commandbox,label type=command}{commands}
\crefname{command}{Commande}{Commandes}

% Exercices
\tcbset{
    exercisebox/.style={
            colframe=exercisecolor!70,
            colback=exercisecolor!2,
            coltitle=exercisecolor!50!black,
            title style={exercisecolor!40},
        %     fonttitle=\sc
        }
}

\newtcbtheorem[use counter=dummy, number within=section]{exercise}{Exercice}{globalbox,exercisebox,label type=exercise}{exercises}
\crefname{exercise}{Exercice}{Exercices}

\newtcbtheorem[use counter=dummy, number within=section]{example}{Exemple}{globalbox,exercisebox,label type=example, leftrule=0pt, rightrule=0pt}{examples}
\crefname{example}{Exemple}{Exemples}

% Exercices de DS
\tcbset{
    testexercisebox/.style={
            colframe=testexercisecolor,
            colback=white,
            coltitle=testexercisecolor!20!black,
            title style={testexercisecolor!50},
        }
}

\newtcbtheorem{testexercise}{Exercice}{globalbox,testexercisebox,label type=testexercise}{testexercises}
\crefname{testexercise}{Exercice}{Exercices}

% Correction d'exercice
% \theoremheaderfont{\sc}\theorembodyfont{\upshape}
% \theoremstyle{nonumberbreak}
% \theoremseparator{}
% \theoremsymbol{\rule{1ex}{1ex}}
\theoremstyle{definition}
\newtheorem*{correction}{Correction}

% \tcolorboxenvironment
% {correction}{
%     blanker,breakable,left=5mm,
%     before skip=10pt,after skip=10pt,
%     borderline west={1mm}{0pt}{exercisecolor}}

\tcolorboxenvironment{correction}{% 'proof' from 'amsthm'
blanker,breakable,left=5mm,
before skip=10pt,after skip=10pt,
borderline west={1mm}{0pt}{exercisecolor}}

% Remarque
\tcbset{
    remarkbox/.style={
            breakable,
            enhanced,
            colback=remarkcolor!2,
            colframe=remarkcolor,
            arc=0pt,
            outer arc=0pt,
            % fonttitle=\bfseries\sffamily\large,
            fonttitle=\bfseries,
            boxrule=1pt,
            colbacktitle=remarkcolor,
            attach boxed title to top left={},
            boxed title style={
                    enhanced,
                    skin=enhancedfirst jigsaw,
                    arc=0pt,
                    bottom=0pt,
                    interior style={fill=remarkcolor}
                }
        }
}

\newtcbtheorem[use counter=dummy, number within=section]{remark}{Remarque}{remarkbox,label type=remark}{remarks}
\crefname{remark}{Remarque}{Remarques}

% Méthode
\tcbset{
    methodbox/.style={
            colback=methodcolor!5,
            colframe=methodcolor,
            colbacktitle=methodcolor,
            attach boxed title to top center={},
            boxed title style={
                    enhanced,
                    skin=enhancedfirst jigsaw,
                    arc=0pt,
                    bottom=0pt,
                    interior style={fill=methodcolor}
                }
        }
}

\newtcbtheorem[use counter=dummy, number within=section]{method}{Méthode}{remarkbox,methodbox, label type=method}{methods}
\crefname{method}{Méthode}{Méthodes}

% Commentaire
\tcbset{
    commentbox/.style={
            colback=commentcolor!5,
            colframe=commentcolor,
            colbacktitle=commentcolor,
            attach boxed title to top right={},
            boxed title style={
                    enhanced,
                    skin=enhancedfirst jigsaw,
                    arc=0pt,
                    bottom=0pt,
                    interior style={fill=commentcolor}
                }
        }
}

\newtcolorbox{note}[1][Commentaire]{remarkbox,commentbox,title=#1}

% Todos
\tcbset{
    todobox/.style={
        %     breakable,
        %     enhanced,
            arc=0pt,
            outer arc=0pt,
            fonttitle=\bfseries,
        %     boxrule=1pt,
        %     width=(\linewidth - 4pt)/3,
                width=0.33\linewidth,
            equal height group=AT,
            before=,
            after=\hfill,
        %     boxed title style={
        %             enhanced,
        %             skin=enhancedfirst jigsaw,
        %             arc=0pt,
        %             bottom=0pt,
        %         }
        }
}

\tcbset{
    bigtodobox/.style={
                enhanced,
                arc=0pt,
                outer arc=0pt,
                fonttitle=\bfseries,
            colback=white,
                % interior hidden,
            colframe=bigtodocolor,
            colbacktitle=bigtodocolor,
            left=0mm,
            right=0mm,
            top=0mm,
            bottom=0mm,
            leftrule=0pt,
            bottomrule=0.8pt,
            toprule=0.8pt,
            rightrule=0pt,
            attach boxed title to top right={},
            boxed title style={
                    enhanced,
                    skin=enhancedfirst jigsaw,
                    arc=0pt,
                    bottom=0pt,
                    interior style={fill=bigtodocolor}
                }
        }
}

\newtcolorbox{bigtodo}[1][Pour s'entraîner]{bigtodobox,title=#1}

\tcbset{
    todolv1box/.style={
            colback=todolv1color!5,
            colframe=todolv1color,
            colbacktitle=todolv1color,
            boxed title style={
                    interior style={fill=todolv1color}
                }
        }
}

\newtcolorbox{todolv1}[1][Niveau 1]{todobox,todolv1box,title=#1}

\tcbset{
    todolv2box/.style={
            colback=todolv2color!5,
            colframe=todolv2color,
            colbacktitle=todolv2color,
            boxed title style={
                    interior style={fill=todolv2color}
                }
        }
}

\newtcolorbox{todolv2}[1][Niveau 2]{todobox,todolv2box,title=#1}

\tcbset{
    todolv3box/.style={
            colback=todolv3color!5,
            colframe=todolv3color,
            colbacktitle=todolv3color,
            boxed title style={
                    interior style={fill=todolv3color}
                }
        }
}

\newtcolorbox{todolv3}[1][Niveau 3]{todobox,todolv3box,title=#1}

% Démonstrations
% \theoremheaderfont{\sc}\theorembodyfont{\upshape}
% \theoremstyle{nonumberbreak}
% \theoremseparator{}
% \theoremsymbol{\rule{1ex}{1ex}}
% \newtheorem{proof}{Démonstration}

% \tcolorboxenvironment
% {proof}{
%     % ’proof’ from ’amsthm’
%     blanker,breakable,left=5mm,
%     before skip=10pt,after skip=10pt,
%     borderline west={1mm}{0pt}{proofcolor}}

\tcolorboxenvironment{proof}{% 'proof' from 'amsthm'
blanker,breakable,left=5mm,
before skip=10pt,after skip=10pt,
borderline west={1mm}{0pt}{proofcolor}}

% Sujet (oral)
% \tcbset{
%     subjectbox/.style={
%             colback=subjectcolor!5,
%             colback=white,
%             colframe=subjectcolor!40,
%             colbacktitle=subjectcolor,
%             coltitle=subjectcolor!20!black,
%             % attach boxed title to top center={yshift=-2.5mm},
%             attach boxed title to top center={},
%             boxrule=1pt,
%             boxed title style={
%                     enhanced,
%                     skin=enhancedfirst jigsaw,
%                     arc=0pt,
%                     bottom=0pt,
%                     % interior style={fill=subjectcolor}
%                     interior style={fill=subjectcolor!10}
%                 }
%         }
% }

\tcbset{
    subjectbox/.style={
            colback=subjectcolor!5,
            colback=white,
            colframe=subjectcolor!40,
            colbacktitle=subjectcolor!40,
            coltitle=subjectcolor!20!black,
            center title,
            leftrule=0.5pt,
    }
}

% \newtcolorbox{kholle}[1][]{remarkbox,khollebox,breakable=true,title={Sujet de khôlle\ifthenelse{\isempty{#1}}{}{ - #1}}}
\newcounter{subject}
% \newtcbtheorem[use counter=subject]{subject}{Sujet}{remarkbox,subjectbox,breakable=true}{subjects}
\newtcbtheorem[use counter=subject]{subject}{Sujet}{globalbox,subjectbox,breakable=true}{subjects}
\newtcbtheorem[number within=subject]{subjectexercise}{Exercice}{globalbox,exercisebox,label type=exercise}{exercises}

\newtcbox{\rating}{enhanced,nobeforeafter,tcbox raise base,boxrule=0.4pt,top=0mm,bottom=0mm,
  right=0mm,left=4mm,arc=1pt,boxsep=2pt,before upper={\vphantom{dlg}},
  colframe=orange!50!black,coltext=orange!25!black,colback=orange!10!white,
  fontlower=\Large,fontupper=\scriptsize\bfseries,
  overlay={\begin{tcbclipinterior}\fill[orange!75!red!50!white] (frame.south west)
  rectangle node[text=white,font=\sffamily\bfseries\tiny,rotate=90] {PTS} ([xshift=4mm]frame.north west);\end{tcbclipinterior}}}

\newenvironment{todo}{
        \begin{bigtodo}
        \newcommand{\nextlevel}{\end{todolv1}\begin{todolv2}\renewcommand{\nextlevel}{\end{todolv2}\begin{todolv3}}}
        \begin{todolv1}
}{
        \end{todolv3}
        \end{bigtodo}
}
% !TeX root = ../manual.tex

\section{Tableaux de variation}

    Voir \href{https://zestedesavoir.com/tutoriels/439/des-tableaux-de-variations-et-de-signes-avec-latex/}{ce tutoriel}.

    \begin{latexcode}
        \begin{tikzpicture}
            \tkzTabInit[color]{$x$ / 1 , $f'(x)$ / 1, $f$ / 2} % Lignes (nom / taille)
            {$0$, $2$, $5$, $+\infty$}
            \tkzTabLine{z, -, d, h, d, +, }
            \tkzTabVar{+ / $13$, -DH / $4$,  D- / $\frac{\pi}{12}$, + / 15 }
            \tkzTabVal{3}{4}{0.5}{$\frac{\sqrt{333}}{2}$}{$7$}
        \end{tikzpicture}
    \end{latexcode}
%%
%% This is file `minted.sty',
%% generated with the docstrip utility.
%%
%% The original source files were:
%%
%% minted.dtx  (with options: `package')
%% Copyright 2013--2017 Geoffrey M. Poore
%% Copyright 2010--2011 Konrad Rudolph
%% 
%% This work may be distributed and/or modified under the
%% conditions of the LaTeX Project Public License, either version 1.3
%% of this license or (at your option) any later version.
%% The latest version of this license is in
%%   http://www.latex-project.org/lppl.txt
%% and version 1.3 or later is part of all distributions of LaTeX
%% version 2005/12/01 or later.
%% 
%% Additionally, the project may be distributed under the terms of the new BSD
%% license.
%% 
%% This work has the LPPL maintenance status `maintained'.
%% 
%% The Current Maintainer of this work is Geoffrey Poore.
%% 
%% This work consists of the files minted.dtx and minted.ins
%% and the derived file minted.sty.
\NeedsTeXFormat{LaTeX2e}
\ProvidesPackage{minted}
    [2017/09/03 v2.5.1dev Yet another Pygments shim for LaTeX]
\RequirePackage{keyval}
\RequirePackage{kvoptions}
\RequirePackage{fvextra}
\RequirePackage{ifthen}
\RequirePackage{calc}
\IfFileExists{shellesc.sty}
 {\RequirePackage{shellesc}
  \@ifpackagelater{shellesc}{2016/04/29}
   {}
   {\protected\def\ShellEscape{\immediate\write18 }}}
 {\protected\def\ShellEscape{\immediate\write18 }}
\RequirePackage{ifplatform}
\RequirePackage{pdftexcmds}
\RequirePackage{etoolbox}
\RequirePackage{xstring}
\RequirePackage{lineno}
\RequirePackage{framed}
\AtEndPreamble{%
  \@ifpackageloaded{color}{}{%
    \@ifpackageloaded{xcolor}{}{\RequirePackage{xcolor}}}%
}
\DeclareVoidOption{chapter}{\def\minted@float@within{chapter}}
\DeclareVoidOption{section}{\def\minted@float@within{section}}
\DeclareBoolOption{newfloat}
\DeclareBoolOption[true]{cache}
\StrSubstitute{\jobname}{ }{_}[\minted@jobname]
\StrSubstitute{\minted@jobname}{*}{_}[\minted@jobname]
\StrSubstitute{\minted@jobname}{"}{}[\minted@jobname]
\StrSubstitute{\minted@jobname}{'}{_}[\minted@jobname]
\newcommand{\minted@cachedir}{\detokenize{_}minted-\minted@jobname}
\let\minted@cachedir@windows\minted@cachedir
\define@key{minted}{cachedir}{%
  \@namedef{minted@cachedir}{#1}%
  \StrSubstitute{\minted@cachedir}{/}{\@backslashchar}[\minted@cachedir@windows]}
\DeclareBoolOption{finalizecache}
\DeclareBoolOption{frozencache}
\let\minted@outputdir\@empty
\let\minted@outputdir@windows\@empty
\define@key{minted}{outputdir}{%
  \@namedef{minted@outputdir}{#1/}%
  \StrSubstitute{\minted@outputdir}{/}%
    {\@backslashchar}[\minted@outputdir@windows]}
\DeclareBoolOption{kpsewhich}
\DeclareBoolOption{langlinenos}
\DeclareBoolOption{draft}
\DeclareComplementaryOption{final}{draft}
\ProcessKeyvalOptions*
\ifthenelse{\boolean{minted@newfloat}}{\RequirePackage{newfloat}}{\RequirePackage{float}}
\ifcsname tikzifexternalizing\endcsname
  \tikzifexternalizing{\minted@drafttrue\minted@cachefalse}{}
\else
  \ifcsname tikzexternalrealjob\endcsname
    \minted@drafttrue
    \minted@cachefalse
  \else
  \fi
\fi
\ifthenelse{\boolean{minted@finalizecache}}%
 {\ifthenelse{\boolean{minted@frozencache}}%
   {\PackageError{minted}%
     {Options "finalizecache" and "frozencache" are not compatible}%
     {Options "finalizecache" and "frozencache" are not compatible}}%
   {}}%
 {}
\ifthenelse{\boolean{minted@cache}}%
 {\ifthenelse{\boolean{minted@frozencache}}%
   {}%
   {\AtEndOfPackage{\ProvideDirectory{\minted@outputdir\minted@cachedir}}}}%
 {}
\newcommand{\minted@input}[1]{%
  \IfFileExists{#1}%
   {\input{#1}}%
   {\PackageError{minted}{Missing Pygments output; \string\inputminted\space
     was^^Jprobably given a file that does not exist--otherwise, you may need
     ^^Jthe outputdir package option, or may be using an incompatible build
     tool,^^Jor may be using frozencache with a missing file}%
    {This could be caused by using -output-directory or -aux-directory
     ^^Jwithout setting minted's outputdir, or by using a build tool that
     ^^Jchanges paths in ways minted cannot detect,
     ^^Jor using frozencache with a missing file.}}%
}
\newcommand{\minted@infile}{\minted@jobname.out.pyg}
\newcommand{\minted@cachelist}{}
\newcommand{\minted@addcachefile}[1]{%
  \expandafter\long\expandafter\gdef\expandafter\minted@cachelist\expandafter{%
    \minted@cachelist,^^J%
    \space\space#1}%
  \expandafter\gdef\csname minted@cached@#1\endcsname{}%
}
\newcommand{\minted@savecachelist}{%
  \ifdefempty{\minted@cachelist}{}{%
    \immediate\write\@mainaux{%
      \string\gdef\string\minted@oldcachelist\string{%
        \minted@cachelist\string}}%
  }%
}
\newcommand{\minted@cleancache}{%
  \ifcsname minted@oldcachelist\endcsname
    \def\do##1{%
      \ifthenelse{\equal{##1}{}}{}{%
        \ifcsname minted@cached@##1\endcsname\else
          \DeleteFile[\minted@outputdir\minted@cachedir]{##1}%
        \fi
      }%
    }%
    \expandafter\docsvlist\expandafter{\minted@oldcachelist}%
  \else
  \fi
}
\ifthenelse{\boolean{minted@draft}}%
 {\AtEndDocument{%
    \ifcsname minted@oldcachelist\endcsname
      \StrSubstitute{\minted@oldcachelist}{,}{,^^J }[\minted@cachelist]
      \minted@savecachelist
    \fi}}%
 {\ifthenelse{\boolean{minted@frozencache}}%
   {\AtEndDocument{%
      \ifcsname minted@oldcachelist\endcsname
        \StrSubstitute{\minted@oldcachelist}{,}{,^^J }[\minted@cachelist]
        \minted@savecachelist
      \fi}}%
   {\AtEndDocument{%
    \minted@savecachelist
    \minted@cleancache}}}%
\ifwindows
  \providecommand{\DeleteFile}[2][]{%
    \ifthenelse{\equal{#1}{}}%
      {\IfFileExists{#2}{\ShellEscape{del #2}}{}}%
      {\IfFileExists{#1/#2}{%
        \StrSubstitute{#1}{/}{\@backslashchar}[\minted@windir]
        \ShellEscape{del \minted@windir\@backslashchar #2}}{}}}
\else
  \providecommand{\DeleteFile}[2][]{%
    \ifthenelse{\equal{#1}{}}%
      {\IfFileExists{#2}{\ShellEscape{rm #2}}{}}%
      {\IfFileExists{#1/#2}{\ShellEscape{rm #1/#2}}{}}}
\fi
\ifwindows
  \newcommand{\ProvideDirectory}[1]{%
    \StrSubstitute{#1}{/}{\@backslashchar}[\minted@windir]
    \ShellEscape{if not exist \minted@windir\space mkdir \minted@windir}}
\else
  \newcommand{\ProvideDirectory}[1]{%
    \ShellEscape{mkdir -p #1}}
\fi
\newboolean{AppExists}
\newread\minted@appexistsfile
\newcommand{\TestAppExists}[1]{
  \ifwindows
    \DeleteFile{\minted@jobname.aex}
    \ShellEscape{for \string^\@percentchar i in (#1.exe #1.bat #1.cmd)
      do set > \minted@jobname.aex <nul: /p
      x=\string^\@percentchar \string~$PATH:i>> \minted@jobname.aex}
    %$ <- balance syntax highlighting
    \immediate\openin\minted@appexistsfile\minted@jobname.aex
    \expandafter\def\expandafter\@tmp@cr\expandafter{\the\endlinechar}
    \endlinechar=-1\relax
    \readline\minted@appexistsfile to \minted@apppathifexists
    \endlinechar=\@tmp@cr
    \ifthenelse{\equal{\minted@apppathifexists}{}}
     {\AppExistsfalse}
     {\AppExiststrue}
    \immediate\closein\minted@appexistsfile
    \DeleteFile{\minted@jobname.aex}
  \else
    \ShellEscape{which #1 && touch \minted@jobname.aex}
    \IfFileExists{\minted@jobname.aex}
      {\AppExiststrue
        \DeleteFile{\minted@jobname.aex}}
      {\AppExistsfalse}
  \fi
}
\newcommand{\minted@optlistcl@g}{}
\newcommand{\minted@optlistcl@g@i}{}
\let\minted@lang\@empty
\newcommand{\minted@optlistcl@lang}{}
\newcommand{\minted@optlistcl@lang@i}{}
\newcommand{\minted@optlistcl@cmd}{}
\newcommand{\minted@optlistfv@g}{}
\newcommand{\minted@optlistfv@g@i}{}
\newcommand{\minted@optlistfv@lang}{}
\newcommand{\minted@optlistfv@lang@i}{}
\newcommand{\minted@optlistfv@cmd}{}
\newcommand{\minted@configlang}[1]{%
  \def\minted@lang{#1}%
  \ifcsname minted@optlistcl@lang\minted@lang\endcsname\else
    \expandafter\gdef\csname minted@optlistcl@lang\minted@lang\endcsname{}%
  \fi
  \ifcsname minted@optlistcl@lang\minted@lang @i\endcsname\else
    \expandafter\gdef\csname minted@optlistcl@lang\minted@lang @i\endcsname{}%
  \fi
  \ifcsname minted@optlistfv@lang\minted@lang\endcsname\else
    \expandafter\gdef\csname minted@optlistfv@lang\minted@lang\endcsname{}%
  \fi
  \ifcsname minted@optlistfv@lang\minted@lang @i\endcsname\else
    \expandafter\gdef\csname minted@optlistfv@lang\minted@lang @i\endcsname{}%
  \fi
}
\newcommand{\minted@addto@optlistcl}[2]{%
  \expandafter\def\expandafter#1\expandafter{#1%
    \detokenize{#2}\space}}
\newcommand{\minted@addto@optlistcl@lang}[2]{%
  \expandafter\let\expandafter\minted@tmp\csname #1\endcsname
  \expandafter\def\expandafter\minted@tmp\expandafter{\minted@tmp%
    \detokenize{#2}\space}%
  \expandafter\let\csname #1\endcsname\minted@tmp}
\newcommand{\minted@def@optcl}[4][]{%
  \ifthenelse{\equal{#1}{}}%
    {\define@key{minted@opt@g}{#2}{%
        \minted@addto@optlistcl{\minted@optlistcl@g}{#3=#4}%
        \@namedef{minted@opt@g:#2}{#4}}%
      \define@key{minted@opt@g@i}{#2}{%
        \minted@addto@optlistcl{\minted@optlistcl@g@i}{#3=#4}%
        \@namedef{minted@opt@g@i:#2}{#4}}%
      \define@key{minted@opt@lang}{#2}{%
        \minted@addto@optlistcl@lang{minted@optlistcl@lang\minted@lang}{#3=#4}%
        \@namedef{minted@opt@lang\minted@lang:#2}{#4}}%
      \define@key{minted@opt@lang@i}{#2}{%
        \minted@addto@optlistcl@lang{%
          minted@optlistcl@lang\minted@lang @i}{#3=#4}%
        \@namedef{minted@opt@lang\minted@lang @i:#2}{#4}}%
      \define@key{minted@opt@cmd}{#2}{%
        \minted@addto@optlistcl{\minted@optlistcl@cmd}{#3=#4}%
        \@namedef{minted@opt@cmd:#2}{#4}}}%
    {\define@key{minted@opt@g}{#2}[#1]{%
        \minted@addto@optlistcl{\minted@optlistcl@g}{#3=#4}%
        \@namedef{minted@opt@g:#2}{#4}}%
      \define@key{minted@opt@g@i}{#2}[#1]{%
        \minted@addto@optlistcl{\minted@optlistcl@g@i}{#3=#4}%
        \@namedef{minted@opt@g@i:#2}{#4}}%
      \define@key{minted@opt@lang}{#2}[#1]{%
        \minted@addto@optlistcl@lang{minted@optlistcl@lang\minted@lang}{#3=#4}%
        \@namedef{minted@opt@lang\minted@lang:#2}{#4}}%
      \define@key{minted@opt@lang@i}{#2}[#1]{%
        \minted@addto@optlistcl@lang{%
          minted@optlistcl@lang\minted@lang @i}{#3=#4}%
        \@namedef{minted@opt@lang\minted@lang @i:#2}{#4}}%
      \define@key{minted@opt@cmd}{#2}[#1]{%
        \minted@addto@optlistcl{\minted@optlistcl@cmd}{#3=#4}%
        \@namedef{minted@opt@cmd:#2}{#4}}}%
}
\edef\minted@hashchar{\string#}
\edef\minted@dollarchar{\string$}
\edef\minted@ampchar{\string&}
\edef\minted@underscorechar{\string_}
\edef\minted@tildechar{\string~}
\edef\minted@leftsquarebracket{\string[}
\edef\minted@rightsquarebracket{\string]}
\newcommand{\minted@escchars}{%
  \let\#\minted@hashchar
  \let\%\@percentchar
  \let\{\@charlb
  \let\}\@charrb
  \let\$\minted@dollarchar
  \let\&\minted@ampchar
  \let\_\minted@underscorechar
  \let\\\@backslashchar
  \let~\minted@tildechar
  \let\~\minted@tildechar
  \let\[\minted@leftsquarebracket
  \let\]\minted@rightsquarebracket
} %$ <- highlighting
\newcommand{\minted@addto@optlistcl@e}[2]{%
  \begingroup
  \minted@escchars
  \xdef\minted@xtmp{#2}%
  \endgroup
  \expandafter\minted@addto@optlistcl@e@i\expandafter{\minted@xtmp}{#1}}
\def\minted@addto@optlistcl@e@i#1#2{%
  \expandafter\def\expandafter#2\expandafter{#2#1\space}}
\newcommand{\minted@addto@optlistcl@lang@e}[2]{%
  \begingroup
  \minted@escchars
  \xdef\minted@xtmp{#2}%
  \endgroup
  \expandafter\minted@addto@optlistcl@lang@e@i\expandafter{\minted@xtmp}{#1}}
\def\minted@addto@optlistcl@lang@e@i#1#2{%
  \expandafter\let\expandafter\minted@tmp\csname #2\endcsname
  \expandafter\def\expandafter\minted@tmp\expandafter{\minted@tmp#1\space}%
  \expandafter\let\csname #2\endcsname\minted@tmp}
\newcommand{\minted@def@optcl@e}[4][]{%
  \ifthenelse{\equal{#1}{}}%
    {\define@key{minted@opt@g}{#2}{%
        \minted@addto@optlistcl@e{\minted@optlistcl@g}{#3=#4}%
        \@namedef{minted@opt@g:#2}{#4}}%
      \define@key{minted@opt@g@i}{#2}{%
        \minted@addto@optlistcl@e{\minted@optlistcl@g@i}{#3=#4}%
        \@namedef{minted@opt@g@i:#2}{#4}}%
      \define@key{minted@opt@lang}{#2}{%
        \minted@addto@optlistcl@lang@e{minted@optlistcl@lang\minted@lang}{#3=#4}%
        \@namedef{minted@opt@lang\minted@lang:#2}{#4}}%
      \define@key{minted@opt@lang@i}{#2}{%
        \minted@addto@optlistcl@lang@e{%
          minted@optlistcl@lang\minted@lang @i}{#3=#4}%
        \@namedef{minted@opt@lang\minted@lang @i:#2}{#4}}%
      \define@key{minted@opt@cmd}{#2}{%
        \minted@addto@optlistcl@e{\minted@optlistcl@cmd}{#3=#4}%
        \@namedef{minted@opt@cmd:#2}{#4}}}%
    {\define@key{minted@opt@g}{#2}[#1]{%
        \minted@addto@optlistcl@e{\minted@optlistcl@g}{#3=#4}%
        \@namedef{minted@opt@g:#2}{#4}}%
      \define@key{minted@opt@g@i}{#2}[#1]{%
        \minted@addto@optlistcl@e{\minted@optlistcl@g@i}{#3=#4}%
        \@namedef{minted@opt@g@i:#2}{#4}}%
      \define@key{minted@opt@lang}{#2}[#1]{%
        \minted@addto@optlistcl@lang@e{minted@optlistcl@lang\minted@lang}{#3=#4}%
        \@namedef{minted@opt@lang\minted@lang:#2}{#4}}%
      \define@key{minted@opt@lang@i}{#2}[#1]{%
        \minted@addto@optlistcl@lang@e{%
          minted@optlistcl@lang\minted@lang @i}{#3=#4}%
        \@namedef{minted@opt@lang\minted@lang @i:#2}{#4}}%
      \define@key{minted@opt@cmd}{#2}[#1]{%
        \minted@addto@optlistcl@e{\minted@optlistcl@cmd}{#3=#4}%
        \@namedef{minted@opt@cmd:#2}{#4}}}%
}
\newcommand{\minted@def@optcl@switch}[2]{%
  \define@booleankey{minted@opt@g}{#1}%
    {\minted@addto@optlistcl{\minted@optlistcl@g}{#2=True}%
      \@namedef{minted@opt@g:#1}{true}}
    {\minted@addto@optlistcl{\minted@optlistcl@g}{#2=False}%
      \@namedef{minted@opt@g:#1}{false}}
  \define@booleankey{minted@opt@g@i}{#1}%
    {\minted@addto@optlistcl{\minted@optlistcl@g@i}{#2=True}%
      \@namedef{minted@opt@g@i:#1}{true}}
    {\minted@addto@optlistcl{\minted@optlistcl@g@i}{#2=False}%
      \@namedef{minted@opt@g@i:#1}{false}}
  \define@booleankey{minted@opt@lang}{#1}%
    {\minted@addto@optlistcl@lang{minted@optlistcl@lang\minted@lang}{#2=True}%
      \@namedef{minted@opt@lang\minted@lang:#1}{true}}
    {\minted@addto@optlistcl@lang{minted@optlistcl@lang\minted@lang}{#2=False}%
      \@namedef{minted@opt@lang\minted@lang:#1}{false}}
  \define@booleankey{minted@opt@lang@i}{#1}%
    {\minted@addto@optlistcl@lang{minted@optlistcl@lang\minted@lang @i}{#2=True}%
      \@namedef{minted@opt@lang\minted@lang @i:#1}{true}}
    {\minted@addto@optlistcl@lang{minted@optlistcl@lang\minted@lang @i}{#2=False}%
      \@namedef{minted@opt@lang\minted@lang @i:#1}{false}}
  \define@booleankey{minted@opt@cmd}{#1}%
      {\minted@addto@optlistcl{\minted@optlistcl@cmd}{#2=True}%
        \@namedef{minted@opt@cmd:#1}{true}}
      {\minted@addto@optlistcl{\minted@optlistcl@cmd}{#2=False}%
        \@namedef{minted@opt@cmd:#1}{false}}
}
\newcommand{\minted@def@optfv}[1]{%
  \define@key{minted@opt@g}{#1}{%
    \expandafter\def\expandafter\minted@optlistfv@g\expandafter{%
      \minted@optlistfv@g#1={##1},}%
    \@namedef{minted@opt@g:#1}{##1}}
  \define@key{minted@opt@g@i}{#1}{%
    \expandafter\def\expandafter\minted@optlistfv@g@i\expandafter{%
      \minted@optlistfv@g@i#1={##1},}%
    \@namedef{minted@opt@g@i:#1}{##1}}
  \define@key{minted@opt@lang}{#1}{%
    \expandafter\let\expandafter\minted@tmp%
      \csname minted@optlistfv@lang\minted@lang\endcsname
    \expandafter\def\expandafter\minted@tmp\expandafter{%
      \minted@tmp#1={##1},}%
    \expandafter\let\csname minted@optlistfv@lang\minted@lang\endcsname%
      \minted@tmp
    \@namedef{minted@opt@lang\minted@lang:#1}{##1}}
  \define@key{minted@opt@lang@i}{#1}{%
    \expandafter\let\expandafter\minted@tmp%
      \csname minted@optlistfv@lang\minted@lang @i\endcsname
    \expandafter\def\expandafter\minted@tmp\expandafter{%
      \minted@tmp#1={##1},}%
    \expandafter\let\csname minted@optlistfv@lang\minted@lang @i\endcsname%
      \minted@tmp
    \@namedef{minted@opt@lang\minted@lang @i:#1}{##1}}
  \define@key{minted@opt@cmd}{#1}{%
    \expandafter\def\expandafter\minted@optlistfv@cmd\expandafter{%
      \minted@optlistfv@cmd#1={##1},}%
    \@namedef{minted@opt@cmd:#1}{##1}}
}
\newcommand{\minted@def@optfv@switch}[1]{%
  \define@booleankey{minted@opt@g}{#1}%
    {\expandafter\def\expandafter\minted@optlistfv@g\expandafter{%
      \minted@optlistfv@g#1=true,}%
     \@namedef{minted@opt@g:#1}{true}}%
    {\expandafter\def\expandafter\minted@optlistfv@g\expandafter{%
      \minted@optlistfv@g#1=false,}%
     \@namedef{minted@opt@g:#1}{false}}%
  \define@booleankey{minted@opt@g@i}{#1}%
    {\expandafter\def\expandafter\minted@optlistfv@g@i\expandafter{%
      \minted@optlistfv@g@i#1=true,}%
     \@namedef{minted@opt@g@i:#1}{true}}%
    {\expandafter\def\expandafter\minted@optlistfv@g@i\expandafter{%
      \minted@optlistfv@g@i#1=false,}%
     \@namedef{minted@opt@g@i:#1}{false}}%
  \define@booleankey{minted@opt@lang}{#1}%
    {\expandafter\let\expandafter\minted@tmp%
        \csname minted@optlistfv@lang\minted@lang\endcsname
      \expandafter\def\expandafter\minted@tmp\expandafter{%
        \minted@tmp#1=true,}%
      \expandafter\let\csname minted@optlistfv@lang\minted@lang\endcsname%
        \minted@tmp
     \@namedef{minted@opt@lang\minted@lang:#1}{true}}%
    {\expandafter\let\expandafter\minted@tmp%
        \csname minted@optlistfv@lang\minted@lang\endcsname
      \expandafter\def\expandafter\minted@tmp\expandafter{%
        \minted@tmp#1=false,}%
      \expandafter\let\csname minted@optlistfv@lang\minted@lang\endcsname%
        \minted@tmp
     \@namedef{minted@opt@lang\minted@lang:#1}{false}}%
  \define@booleankey{minted@opt@lang@i}{#1}%
    {\expandafter\let\expandafter\minted@tmp%
        \csname minted@optlistfv@lang\minted@lang @i\endcsname
      \expandafter\def\expandafter\minted@tmp\expandafter{%
        \minted@tmp#1=true,}%
      \expandafter\let\csname minted@optlistfv@lang\minted@lang @i\endcsname%
        \minted@tmp
     \@namedef{minted@opt@lang\minted@lang @i:#1}{true}}%
    {\expandafter\let\expandafter\minted@tmp%
        \csname minted@optlistfv@lang\minted@lang @i\endcsname
      \expandafter\def\expandafter\minted@tmp\expandafter{%
        \minted@tmp#1=false,}%
      \expandafter\let\csname minted@optlistfv@lang\minted@lang @i\endcsname%
        \minted@tmp
     \@namedef{minted@opt@lang\minted@lang @i:#1}{false}}%
  \define@booleankey{minted@opt@cmd}{#1}%
    {\expandafter\def\expandafter\minted@optlistfv@cmd\expandafter{%
      \minted@optlistfv@cmd#1=true,}%
     \@namedef{minted@opt@cmd:#1}{true}}%
    {\expandafter\def\expandafter\minted@optlistfv@cmd\expandafter{%
      \minted@optlistfv@cmd#1=false,}%
     \@namedef{minted@opt@cmd:#1}{false}}%
}
\newboolean{minted@isinline}
\newcommand{\minted@fvset}{%
  \expandafter\fvset\expandafter{\minted@optlistfv@g}%
  \expandafter\let\expandafter\minted@tmp%
    \csname minted@optlistfv@lang\minted@lang\endcsname
  \expandafter\fvset\expandafter{\minted@tmp}%
  \ifthenelse{\boolean{minted@isinline}}%
   {\expandafter\fvset\expandafter{\minted@optlistfv@g@i}%
    \expandafter\let\expandafter\minted@tmp%
      \csname minted@optlistfv@lang\minted@lang @i\endcsname
    \expandafter\fvset\expandafter{\minted@tmp}}%
   {}%
  \expandafter\fvset\expandafter{\minted@optlistfv@cmd}%
}
\newcommand{\minted@def@opt}[2][]{%
  \define@key{minted@opt@g}{#2}{%
    \@namedef{minted@opt@g:#2}{##1}}
  \define@key{minted@opt@g@i}{#2}{%
    \@namedef{minted@opt@g@i:#2}{##1}}
  \define@key{minted@opt@lang}{#2}{%
    \@namedef{minted@opt@lang\minted@lang:#2}{##1}}
  \define@key{minted@opt@lang@i}{#2}{%
    \@namedef{minted@opt@lang\minted@lang @i:#2}{##1}}
  \define@key{minted@opt@cmd}{#2}{%
    \@namedef{minted@opt@cmd:#2}{##1}}
  \ifstrempty{#1}{}{\@namedef{minted@opt@g:#2}{#1}}%
}
\newcommand{\minted@checkstyle}[1]{%
  \ifcsname minted@styleloaded@\ifstrempty{#1}{default-pyg-prefix}{#1}\endcsname\else
    \ifstrempty{#1}{}{\ifcsname PYG\endcsname\else\minted@checkstyle{}\fi}%
    \expandafter\gdef%
      \csname minted@styleloaded@\ifstrempty{#1}{default-pyg-prefix}{#1}\endcsname{}%
    \ifthenelse{\boolean{minted@cache}}%
     {\IfFileExists
       {\minted@outputdir\minted@cachedir/\ifstrempty{#1}{default-pyg-prefix}{#1}.pygstyle}%
       {}%
       {%
        \ifthenelse{\boolean{minted@frozencache}}%
         {\PackageError{minted}%
           {Missing style definition for #1 with frozencache}%
           {Missing style definition for #1 with frozencache}}%
         {\ifwindows
            \ShellEscape{%
              \MintedPygmentize\space -S \ifstrempty{#1}{default}{#1} -f latex
              -P commandprefix=PYG#1
              > \minted@outputdir@windows\minted@cachedir@windows\@backslashchar%
                   \ifstrempty{#1}{default-pyg-prefix}{#1}.pygstyle}%
          \else
            \ShellEscape{%
              \MintedPygmentize\space -S \ifstrempty{#1}{default}{#1} -f latex
              -P commandprefix=PYG#1
              > \minted@outputdir\minted@cachedir/%
                   \ifstrempty{#1}{default-pyg-prefix}{#1}.pygstyle}%
          \fi}%
        }%
        \begingroup
        \let\def\gdef
        \catcode\string``=12
        \catcode`\_=11
        \catcode`\-=11
        \catcode`\%=14
        \endlinechar=-1\relax
        \minted@input{%
          \minted@outputdir\minted@cachedir/\ifstrempty{#1}{default-pyg-prefix}{#1}.pygstyle}%
        \endgroup
        \minted@addcachefile{\ifstrempty{#1}{default-pyg-prefix}{#1}.pygstyle}}%
     {%
        \ifwindows
          \ShellEscape{%
            \MintedPygmentize\space -S \ifstrempty{#1}{default}{#1} -f latex
            -P commandprefix=PYG#1 > \minted@outputdir@windows\minted@jobname.out.pyg}%
        \else
          \ShellEscape{%
            \MintedPygmentize\space -S \ifstrempty{#1}{default}{#1} -f latex
            -P commandprefix=PYG#1 > \minted@outputdir\minted@jobname.out.pyg}%
        \fi
        \begingroup
        \let\def\gdef
        \catcode\string``=12
        \catcode`\_=11
        \catcode`\-=11
        \catcode`\%=14
        \endlinechar=-1\relax
        \minted@input{\minted@outputdir\minted@jobname.out.pyg}%
        \endgroup}%
    \ifstrempty{#1}{\minted@patch@PYGZsq}{}%
  \fi
}
\ifthenelse{\boolean{minted@draft}}{\renewcommand{\minted@checkstyle}[1]{}}{}
\newcommand{\minted@patch@PYGZsq}{%
  \ifcsname PYGZsq\endcsname
    \expandafter\ifdefstring\expandafter{\csname PYGZsq\endcsname}{\char`\'}%
     {\minted@patch@PYGZsq@i}%
     {}%
  \fi
}
\begingroup
\catcode`\'=\active
\gdef\minted@patch@PYGZsq@i{\gdef\PYGZsq{'}}
\endgroup
\ifthenelse{\boolean{minted@draft}}{}{\AtBeginDocument{\minted@patch@PYGZsq}}
\newcommand{\minted@def@opt@switch}[2][false]{%
  \define@booleankey{minted@opt@g}{#2}%
    {\@namedef{minted@opt@g:#2}{true}}%
    {\@namedef{minted@opt@g:#2}{false}}
  \define@booleankey{minted@opt@g@i}{#2}%
    {\@namedef{minted@opt@g@i:#2}{true}}%
    {\@namedef{minted@opt@g@i:#2}{false}}
  \define@booleankey{minted@opt@lang}{#2}%
    {\@namedef{minted@opt@lang\minted@lang:#2}{true}}%
    {\@namedef{minted@opt@lang\minted@lang:#2}{false}}
  \define@booleankey{minted@opt@lang@i}{#2}%
    {\@namedef{minted@opt@lang\minted@lang @i:#2}{true}}%
    {\@namedef{minted@opt@lang\minted@lang @i:#2}{false}}
  \define@booleankey{minted@opt@cmd}{#2}%
    {\@namedef{minted@opt@cmd:#2}{true}}%
    {\@namedef{minted@opt@cmd:#2}{false}}%
  \@namedef{minted@opt@g:#2}{#1}%
}
\def\minted@get@opt#1#2{%
  \ifcsname minted@opt@cmd:#1\endcsname
    \csname minted@opt@cmd:#1\endcsname
  \else
    \ifminted@isinline
      \ifcsname minted@opt@lang\minted@lang @i:#1\endcsname
        \csname minted@opt@lang\minted@lang @i:#1\endcsname
      \else
        \ifcsname minted@opt@g@i:#1\endcsname
          \csname minted@opt@g@i:#1\endcsname
        \else
          \ifcsname minted@opt@lang\minted@lang:#1\endcsname
            \csname minted@opt@lang\minted@lang:#1\endcsname
          \else
            \ifcsname minted@opt@g:#1\endcsname
              \csname minted@opt@g:#1\endcsname
            \else
              #2%
            \fi
          \fi
        \fi
      \fi
    \else
      \ifcsname minted@opt@lang\minted@lang:#1\endcsname
        \csname minted@opt@lang\minted@lang:#1\endcsname
      \else
        \ifcsname minted@opt@g:#1\endcsname
          \csname minted@opt@g:#1\endcsname
        \else
          #2%
        \fi
      \fi
    \fi
  \fi
}%
\minted@def@optcl{encoding}{-P encoding}{#1}
\minted@def@optcl{outencoding}{-P outencoding}{#1}
\minted@def@optcl@e{escapeinside}{-P "escapeinside}{#1"}
\minted@def@optcl@switch{stripnl}{-P stripnl}
\minted@def@optcl@switch{stripall}{-P stripall}
\minted@def@optcl@switch{python3}{-P python3}
\minted@def@optcl@switch{funcnamehighlighting}{-P funcnamehighlighting}
\minted@def@optcl@switch{startinline}{-P startinline}
\ifthenelse{\boolean{minted@draft}}%
  {\minted@def@optfv{gobble}}%
  {\minted@def@optcl{gobble}{-F gobble:n}{#1}}
\minted@def@optcl{codetagify}{-F codetagify:codetags}{#1}
\minted@def@optcl{keywordcase}{-F keywordcase:case}{#1}
\minted@def@optcl@switch{texcl}{-P texcomments}
\minted@def@optcl@switch{texcomments}{-P texcomments}
\minted@def@optcl@switch{mathescape}{-P mathescape}
\minted@def@optfv@switch{linenos}
\minted@def@opt{style}
\minted@def@optfv{frame}
\minted@def@optfv{framesep}
\minted@def@optfv{framerule}
\minted@def@optfv{rulecolor}
\minted@def@optfv{numbersep}
\minted@def@optfv{numbers}
\minted@def@optfv{firstnumber}
\minted@def@optfv{stepnumber}
\minted@def@optfv{firstline}
\minted@def@optfv{lastline}
\minted@def@optfv{baselinestretch}
\minted@def@optfv{xleftmargin}
\minted@def@optfv{xrightmargin}
\minted@def@optfv{fillcolor}
\minted@def@optfv{tabsize}
\minted@def@optfv{fontfamily}
\minted@def@optfv{fontsize}
\minted@def@optfv{fontshape}
\minted@def@optfv{fontseries}
\minted@def@optfv{formatcom}
\minted@def@optfv{label}
\minted@def@optfv{labelposition}
\minted@def@optfv{highlightlines}
\minted@def@optfv{highlightcolor}
\minted@def@optfv{space}
\minted@def@optfv{spacecolor}
\minted@def@optfv{tab}
\minted@def@optfv{tabcolor}
\minted@def@optfv{highlightcolor}
\minted@def@optfv@switch{beameroverlays}
\minted@def@optfv@switch{curlyquotes}
\minted@def@optfv@switch{numberfirstline}
\minted@def@optfv@switch{numberblanklines}
\minted@def@optfv@switch{stepnumberfromfirst}
\minted@def@optfv@switch{stepnumberoffsetvalues}
\minted@def@optfv@switch{showspaces}
\minted@def@optfv@switch{resetmargins}
\minted@def@optfv@switch{samepage}
\minted@def@optfv@switch{showtabs}
\minted@def@optfv@switch{obeytabs}
\minted@def@optfv@switch{breaklines}
\minted@def@optfv@switch{breakbytoken}
\minted@def@optfv@switch{breakbytokenanywhere}
\minted@def@optfv{breakindent}
\minted@def@optfv{breakindentnchars}
\minted@def@optfv@switch{breakautoindent}
\minted@def@optfv{breaksymbol}
\minted@def@optfv{breaksymbolsep}
\minted@def@optfv{breaksymbolsepnchars}
\minted@def@optfv{breaksymbolindent}
\minted@def@optfv{breaksymbolindentnchars}
\minted@def@optfv{breaksymbolleft}
\minted@def@optfv{breaksymbolsepleft}
\minted@def@optfv{breaksymbolsepleftnchars}
\minted@def@optfv{breaksymbolindentleft}
\minted@def@optfv{breaksymbolindentleftnchars}
\minted@def@optfv{breaksymbolright}
\minted@def@optfv{breaksymbolsepright}
\minted@def@optfv{breaksymbolseprightnchars}
\minted@def@optfv{breaksymbolindentright}
\minted@def@optfv{breaksymbolindentrightnchars}
\minted@def@optfv{breakbefore}
\minted@def@optfv{breakbeforesymbolpre}
\minted@def@optfv{breakbeforesymbolpost}
\minted@def@optfv@switch{breakbeforegroup}
\minted@def@optfv{breakafter}
\minted@def@optfv@switch{breakaftergroup}
\minted@def@optfv{breakaftersymbolpre}
\minted@def@optfv{breakaftersymbolpost}
\minted@def@optfv@switch{breakanywhere}
\minted@def@optfv{breakanywheresymbolpre}
\minted@def@optfv{breakanywheresymbolpost}
\minted@def@opt{bgcolor}
\minted@def@opt@switch{autogobble}
\newcommand{\minted@encoding}{\minted@get@opt{encoding}{UTF8}}
\newenvironment{minted@snugshade*}[1]{%
  \def\FrameCommand##1{\hskip\@totalleftmargin
    \colorbox{#1}{##1}%
    \hskip-\linewidth \hskip-\@totalleftmargin \hskip\columnwidth}%
  \MakeFramed{\advance\hsize-\width
    \@totalleftmargin\z@ \linewidth\hsize
    \advance\labelsep\fboxsep
    \@setminipage}%
 }{\par\unskip\@minipagefalse\endMakeFramed}
\newsavebox{\minted@bgbox}
\newenvironment{minted@colorbg}[1]{%
  \setlength{\OuterFrameSep}{0pt}%
  \let\minted@tmp\FV@NumberSep
  \edef\FV@NumberSep{%
    \the\numexpr\dimexpr\minted@tmp+\number\fboxsep\relax sp\relax}%
  \medskip
  \begin{minted@snugshade*}{#1}}
 {\end{minted@snugshade*}%
  \medskip\noindent}
\newwrite\minted@code
\newcommand{\minted@savecode}[1]{
  \immediate\openout\minted@code\minted@jobname.pyg\relax
  \immediate\write\minted@code{\expandafter\detokenize\expandafter{#1}}%
  \immediate\closeout\minted@code}
\newcounter{minted@FancyVerbLineTemp}
\newcommand{\minted@write@detok}[1]{%
  \immediate\write\FV@OutFile{\detokenize{#1}}}
\newcommand{\minted@FVB@VerbatimOut}[1]{%
  \setcounter{minted@FancyVerbLineTemp}{\value{FancyVerbLine}}%
  \@bsphack
  \begingroup
    \FV@UseKeyValues
    \FV@DefineWhiteSpace
    \def\FV@Space{\space}%
    \FV@DefineTabOut
    \let\FV@ProcessLine\minted@write@detok
    \immediate\openout\FV@OutFile #1\relax
    \let\FV@FontScanPrep\relax
    \let\@noligs\relax
    \FV@Scan}
\newcommand{\minted@FVE@VerbatimOut}{%
  \immediate\closeout\FV@OutFile\endgroup\@esphack
  \setcounter{FancyVerbLine}{\value{minted@FancyVerbLineTemp}}}%
\ifcsname MintedPygmentize\endcsname\else
  \newcommand{\MintedPygmentize}{pygmentize}
\fi
\newcounter{minted@pygmentizecounter}
\newcommand{\minted@pygmentize}[2][\minted@outputdir\minted@jobname.pyg]{%
  \minted@checkstyle{\minted@get@opt{style}{default}}%
  \stepcounter{minted@pygmentizecounter}%
  \ifthenelse{\equal{\minted@get@opt{autogobble}{false}}{true}}%
    {\def\minted@codefile{\minted@outputdir\minted@jobname.pyg}}%
    {\def\minted@codefile{#1}}%
  \ifthenelse{\boolean{minted@isinline}}%
    {\def\minted@optlistcl@inlines{%
      \minted@optlistcl@g@i
      \csname minted@optlistcl@lang\minted@lang @i\endcsname}}%
    {\let\minted@optlistcl@inlines\@empty}%
  \def\minted@cmd{%
    \ifminted@kpsewhich
      \ifwindows
        \detokenize{for /f "usebackq tokens=*"}\space\@percentchar\detokenize{a in (`kpsewhich}\space\minted@codefile\detokenize{`) do}\space
      \fi
    \fi
    \MintedPygmentize\space -l #2
    -f latex -P commandprefix=PYG -F tokenmerge
    \minted@optlistcl@g \csname minted@optlistcl@lang\minted@lang\endcsname
    \minted@optlistcl@inlines
    \minted@optlistcl@cmd -o \minted@outputdir\minted@infile\space
    \ifminted@kpsewhich
      \ifwindows
        \@percentchar\detokenize{a}%
      \else
        \detokenize{`}kpsewhich \minted@codefile\space
          \detokenize{||} \minted@codefile\detokenize{`}%
      \fi
    \else
      \minted@codefile
    \fi}%
  % For debugging, uncomment: %%%%
  % \immediate\typeout{\minted@cmd}%
  % %%%%
  \ifthenelse{\boolean{minted@cache}}%
    {%
      \ifminted@frozencache
      \else
        \ifx\XeTeXinterchartoks\minted@undefined
          \ifthenelse{\equal{\minted@get@opt{autogobble}{false}}{true}}%
            {\edef\minted@hash{\pdf@filemdfivesum{#1}%
              \pdf@mdfivesum{\minted@cmd autogobble(\ifx\FancyVerbStartNum\z@ 0\else\FancyVerbStartNum\fi-\ifx\FancyVerbStopNum\z@ 0\else\FancyVerbStopNum\fi)}}}%
            {\edef\minted@hash{\pdf@filemdfivesum{#1}%
              \pdf@mdfivesum{\minted@cmd}}}%
        \else
          \ifx\mdfivesum\minted@undefined
            \immediate\openout\minted@code\minted@jobname.mintedcmd\relax
            \immediate\write\minted@code{\minted@cmd}%
            \ifthenelse{\equal{\minted@get@opt{autogobble}{false}}{true}}%
              {\immediate\write\minted@code{autogobble(\ifx\FancyVerbStartNum\z@ 0\else\FancyVerbStartNum\fi-\ifx\FancyVerbStopNum\z@ 0\else\FancyVerbStopNum\fi)}}{}%
            \immediate\closeout\minted@code
            \edef\minted@argone@esc{#1}%
            \StrSubstitute{\minted@argone@esc}{\@backslashchar}{\@backslashchar\@backslashchar}[\minted@argone@esc]%
            \StrSubstitute{\minted@argone@esc}{"}{\@backslashchar"}[\minted@argone@esc]%
            \edef\minted@tmpfname@esc{\minted@outputdir\minted@jobname}%
            \StrSubstitute{\minted@tmpfname@esc}{\@backslashchar}{\@backslashchar\@backslashchar}[\minted@tmpfname@esc]%
            \StrSubstitute{\minted@tmpfname@esc}{"}{\@backslashchar"}[\minted@tmpfname@esc]%
            %Cheating a little here by using ASCII codes to write `{` and `}`
            %in the Python code
            \def\minted@hashcmd{%
              \detokenize{python -c "import hashlib; import os;
                hasher = hashlib.sha1();
                f = open(os.path.expanduser(os.path.expandvars(\"}\minted@tmpfname@esc.mintedcmd\detokenize{\")), \"rb\");
                hasher.update(f.read());
                f.close();
                f = open(os.path.expanduser(os.path.expandvars(\"}\minted@argone@esc\detokenize{\")), \"rb\");
                hasher.update(f.read());
                f.close();
                f = open(os.path.expanduser(os.path.expandvars(\"}\minted@tmpfname@esc.mintedmd5\detokenize{\")), \"w\");
                macro = \"\\edef\\minted@hash\" + chr(123) + hasher.hexdigest() + chr(125) + \"\";
                f.write(\"\\makeatletter\" + macro + \"\\makeatother\\endinput\n\");
                f.close();"}}%
            \ShellEscape{\minted@hashcmd}%
            \minted@input{\minted@outputdir\minted@jobname.mintedmd5}%
          \else
            \ifthenelse{\equal{\minted@get@opt{autogobble}{false}}{true}}%
             {\edef\minted@hash{\mdfivesum file {#1}%
                \mdfivesum{\minted@cmd autogobble(\ifx\FancyVerbStartNum\z@ 0\else\FancyVerbStartNum\fi-\ifx\FancyVerbStopNum\z@ 0\else\FancyVerbStopNum\fi)}}}%
             {\edef\minted@hash{\mdfivesum file {#1}%
                \mdfivesum{\minted@cmd}}}%
          \fi
        \fi
        \edef\minted@infile{\minted@cachedir/\minted@hash.pygtex}%
        \IfFileExists{\minted@infile}{}{%
          \ifthenelse{\equal{\minted@get@opt{autogobble}{false}}{true}}{%
            \minted@autogobble{#1}}{}%
          \ShellEscape{\minted@cmd}}%
      \fi
      \ifthenelse{\boolean{minted@finalizecache}}%
       {%
          \edef\minted@cachefilename{listing\arabic{minted@pygmentizecounter}.pygtex}%
          \edef\minted@actualinfile{\minted@cachedir/\minted@cachefilename}%
          \ifwindows
            \StrSubstitute{\minted@infile}{/}{\@backslashchar}[\minted@infile@windows]
            \StrSubstitute{\minted@actualinfile}{/}{\@backslashchar}[\minted@actualinfile@windows]
            \ShellEscape{move /y \minted@outputdir\minted@infile@windows\space\minted@outputdir\minted@actualinfile@windows}%
          \else
            \ShellEscape{mv -f \minted@outputdir\minted@infile\space\minted@outputdir\minted@actualinfile}%
          \fi
          \let\minted@infile\minted@actualinfile
          \expandafter\minted@addcachefile\expandafter{\minted@cachefilename}%
       }%
       {\ifthenelse{\boolean{minted@frozencache}}%
         {%
            \edef\minted@cachefilename{listing\arabic{minted@pygmentizecounter}.pygtex}%
            \edef\minted@infile{\minted@cachedir/\minted@cachefilename}%
            \expandafter\minted@addcachefile\expandafter{\minted@cachefilename}}%
         {\expandafter\minted@addcachefile\expandafter{\minted@hash.pygtex}}%
       }%
      \minted@inputpyg}%
    {%
      \ifthenelse{\equal{\minted@get@opt{autogobble}{false}}{true}}{%
        \minted@autogobble{#1}}{}%
      \ShellEscape{\minted@cmd}%
      \minted@inputpyg}%
}
\def\minted@autogobble#1{%
  \edef\minted@argone@esc{#1}%
  \StrSubstitute{\minted@argone@esc}{\@backslashchar}{\@backslashchar\@backslashchar}[\minted@argone@esc]%
  \StrSubstitute{\minted@argone@esc}{"}{\@backslashchar"}[\minted@argone@esc]%
  \edef\minted@tmpfname@esc{\minted@outputdir\minted@jobname}%
  \StrSubstitute{\minted@tmpfname@esc}{\@backslashchar}{\@backslashchar\@backslashchar}[\minted@tmpfname@esc]%
  \StrSubstitute{\minted@tmpfname@esc}{"}{\@backslashchar"}[\minted@tmpfname@esc]%
  %Need a version of open() that supports encoding under Python 2
  \edef\minted@autogobblecmd{%
    \ifminted@kpsewhich
      \ifwindows
        \detokenize{for /f "usebackq tokens=*" }\@percentchar\detokenize{a in (`kpsewhich} #1\detokenize{`) do}\space
      \fi
    \fi
    \detokenize{python -c "import sys; import os;
    import textwrap;
    from io import open;
    fname = }%
      \ifminted@kpsewhich
        \detokenize{sys.argv[1];}\space%
      \else
        \detokenize{os.path.expanduser(os.path.expandvars(\"}\minted@argone@esc\detokenize{\"));}\space%
      \fi
    \detokenize{f = open(fname, \"r\", encoding=\"}\minted@encoding\detokenize{\") if os.path.isfile(fname) else None;
    t = f.readlines() if f is not None else None;
    t_opt = t if t is not None else [];
    f.close() if f is not None else None;
    tmpfname = os.path.expanduser(os.path.expandvars(\"}\minted@tmpfname@esc.pyg\detokenize{\"));
    f = open(tmpfname, \"w\", encoding=\"}\minted@encoding\detokenize{\") if t is not None else None;
    fvstartnum = }\ifx\FancyVerbStartNum\z@ 0\else\FancyVerbStartNum\fi\detokenize{;
    fvstopnum = }\ifx\FancyVerbStopNum\z@ 0\else\FancyVerbStopNum\fi\detokenize{;
    s = fvstartnum-1 if fvstartnum != 0 else 0;
    e = fvstopnum if fvstopnum != 0 else len(t_opt);
    [f.write(textwrap.dedent(\"\".join(x))) for x in (t_opt[0:s], t_opt[s:e], t_opt[e:]) if x and t is not None];
    f.close() if t is not None else os.remove(tmpfname);"}%
    \ifminted@kpsewhich
      \ifwindows
        \space\@percentchar\detokenize{a}%
      \else
        \space\detokenize{`}kpsewhich #1\space\detokenize{||} #1\detokenize{`}%
      \fi
    \fi
  }%
  \ShellEscape{\minted@autogobblecmd}%
}
\newcommand{\minted@inputpyg}{%
  \expandafter\let\expandafter\minted@PYGstyle%
    \csname PYG\minted@get@opt{style}{default}\endcsname
  \VerbatimPygments{\PYG}{\minted@PYGstyle}%
  \ifthenelse{\boolean{minted@isinline}}%
   {\ifthenelse{\equal{\minted@get@opt{breaklines}{false}}{true}}%
    {\let\FV@BeginVBox\relax
     \let\FV@EndVBox\relax
     \def\FV@BProcessLine##1{\FancyVerbFormatLine{##1}}%
     \minted@inputpyg@inline}%
    {\minted@inputpyg@inline}}%
   {\minted@inputpyg@block}%
}
\def\minted@inputpyg@inline{%
  \ifthenelse{\equal{\minted@get@opt{bgcolor}{}}{}}%
   {\minted@input{\minted@outputdir\minted@infile}}%
   {\colorbox{\minted@get@opt{bgcolor}{}}{%
      \minted@input{\minted@outputdir\minted@infile}}}%
}
\def\minted@inputpyg@block{%
  \ifthenelse{\equal{\minted@get@opt{bgcolor}{}}{}}%
   {\minted@input{\minted@outputdir\minted@infile}}%
   {\begin{minted@colorbg}{\minted@get@opt{bgcolor}{}}%
    \minted@input{\minted@outputdir\minted@infile}%
    \end{minted@colorbg}}}
\newcommand{\minted@langlinenoson}{%
  \ifcsname c@minted@lang\minted@lang\endcsname\else
    \newcounter{minted@lang\minted@lang}%
  \fi
  \setcounter{minted@FancyVerbLineTemp}{\value{FancyVerbLine}}%
  \setcounter{FancyVerbLine}{\value{minted@lang\minted@lang}}%
}
\newcommand{\minted@langlinenosoff}{%
  \setcounter{minted@lang\minted@lang}{\value{FancyVerbLine}}%
  \setcounter{FancyVerbLine}{\value{minted@FancyVerbLineTemp}}%
}
\ifthenelse{\boolean{minted@langlinenos}}{}{%
  \let\minted@langlinenoson\relax
  \let\minted@langlinenosoff\relax
}
\newcommand{\setminted}[2][]{%
  \ifthenelse{\equal{#1}{}}%
    {\setkeys{minted@opt@g}{#2}}%
    {\minted@configlang{#1}%
      \setkeys{minted@opt@lang}{#2}}}
\newcommand{\setmintedinline}[2][]{%
  \ifthenelse{\equal{#1}{}}%
    {\setkeys{minted@opt@g@i}{#2}}%
    {\minted@configlang{#1}%
      \setkeys{minted@opt@lang@i}{#2}}}
\setmintedinline[php]{startinline=true}
\setminted{tabcolor=black}
\newcommand{\usemintedstyle}[2][]{\setminted[#1]{style=#2}}
\begingroup
\catcode`\ =\active
\catcode`\^^I=\active
\gdef\minted@defwhitespace@retok{\def {\noexpand\FV@Space}\def^^I{\noexpand\FV@Tab}}%
\endgroup
\newcommand{\minted@writecmdcode}[1]{%
  \immediate\openout\minted@code\minted@jobname.pyg\relax
  \immediate\write\minted@code{\detokenize{#1}}%
  \immediate\closeout\minted@code}
\newrobustcmd{\mintinline}[2][]{%
  \begingroup
  \setboolean{minted@isinline}{true}%
  \minted@configlang{#2}%
  \setkeys{minted@opt@cmd}{#1}%
  \minted@fvset
  \begingroup
  \let\do\@makeother\dospecials
  \catcode`\{=1
  \catcode`\}=2
  \catcode`\^^I=\active
  \@ifnextchar\bgroup
    {\minted@inline@iii}%
    {\catcode`\{=12\catcode`\}=12
      \minted@inline@i}}
\def\minted@inline@i#1{%
  \endgroup
  \def\minted@inline@ii##1#1{%
    \minted@inline@iii{##1}}%
  \begingroup
  \let\do\@makeother\dospecials
  \catcode`\^^I=\active
  \minted@inline@ii}
\ifthenelse{\boolean{minted@draft}}%
  {\newcommand{\minted@inline@iii}[1]{%
    \endgroup
    \begingroup
    \minted@defwhitespace@retok
    \everyeof{\noexpand}%
    \endlinechar-1\relax
    \let\do\@makeother\dospecials
    \catcode`\ =\active
    \catcode`\^^I=\active
    \xdef\minted@tmp{\scantokens{#1}}%
    \endgroup
    \let\FV@Line\minted@tmp
    \def\FV@SV@minted@tmp{%
      \FV@Gobble
      \expandafter\FV@ProcessLine\expandafter{\FV@Line}}%
    \ifthenelse{\equal{\minted@get@opt{breaklines}{false}}{true}}%
     {\let\FV@BeginVBox\relax
      \let\FV@EndVBox\relax
      \def\FV@BProcessLine##1{\FancyVerbFormatLine{##1}}%
      \BUseVerbatim{minted@tmp}}%
     {\BUseVerbatim{minted@tmp}}%
    \endgroup}}%
  {\newcommand{\minted@inline@iii}[1]{%
    \endgroup
    \minted@writecmdcode{#1}%
    \RecustomVerbatimEnvironment{Verbatim}{BVerbatim}{}%
    \setcounter{minted@FancyVerbLineTemp}{\value{FancyVerbLine}}%
    \minted@pygmentize{\minted@lang}%
    \setcounter{FancyVerbLine}{\value{minted@FancyVerbLineTemp}}%
    \endgroup}}
\newrobustcmd{\mint}[2][]{%
  \begingroup
  \minted@configlang{#2}%
  \setkeys{minted@opt@cmd}{#1}%
  \minted@fvset
  \begingroup
  \let\do\@makeother\dospecials
  \catcode`\{=1
  \catcode`\}=2
  \catcode`\^^I=\active
  \@ifnextchar\bgroup
    {\mint@iii}%
    {\catcode`\{=12\catcode`\}=12
      \mint@i}}
\def\mint@i#1{%
  \endgroup
  \def\mint@ii##1#1{%
    \mint@iii{##1}}%
  \begingroup
  \let\do\@makeother\dospecials
  \catcode`\^^I=\active
  \mint@ii}
\ifthenelse{\boolean{minted@draft}}%
  {\newcommand{\mint@iii}[1]{%
    \endgroup
    \begingroup
    \minted@defwhitespace@retok
    \everyeof{\noexpand}%
    \endlinechar-1\relax
    \let\do\@makeother\dospecials
    \catcode`\ =\active
    \catcode`\^^I=\active
    \xdef\minted@tmp{\scantokens{#1}}%
    \endgroup
    \let\FV@Line\minted@tmp
    \def\FV@SV@minted@tmp{%
      \FV@CodeLineNo=1\FV@StepLineNo
      \FV@Gobble
      \expandafter\FV@ProcessLine\expandafter{\FV@Line}}%
    \minted@langlinenoson
    \UseVerbatim{minted@tmp}%
    \minted@langlinenosoff
    \endgroup}}%
  {\newcommand{\mint@iii}[1]{%
    \endgroup
    \minted@writecmdcode{#1}%
    \minted@langlinenoson
    \minted@pygmentize{\minted@lang}%
    \minted@langlinenosoff
    \endgroup}}
\ifthenelse{\boolean{minted@draft}}%
  {\newenvironment{minted}[2][]
    {\VerbatimEnvironment
      \minted@configlang{#2}%
      \setkeys{minted@opt@cmd}{#1}%
      \minted@fvset
      \minted@langlinenoson
      \begin{Verbatim}}%
    {\end{Verbatim}%
      \minted@langlinenosoff}}%
  {\newenvironment{minted}[2][]
    {\VerbatimEnvironment
      \let\FVB@VerbatimOut\minted@FVB@VerbatimOut
      \let\FVE@VerbatimOut\minted@FVE@VerbatimOut
      \minted@configlang{#2}%
      \setkeys{minted@opt@cmd}{#1}%
      \minted@fvset
      \begin{VerbatimOut}[codes={\catcode`\^^I=12},firstline,lastline]{\minted@jobname.pyg}}%
    {\end{VerbatimOut}%
        \minted@langlinenoson
        \minted@pygmentize{\minted@lang}%
        \minted@langlinenosoff}}
\ifthenelse{\boolean{minted@draft}}%
  {\newcommand{\inputminted}[3][]{%
    \begingroup
    \minted@configlang{#2}%
    \setkeys{minted@opt@cmd}{#1}%
    \minted@fvset
    \VerbatimInput{#3}%
    \endgroup}}%
  {\newcommand{\inputminted}[3][]{%
    \begingroup
    \minted@configlang{#2}%
    \setkeys{minted@opt@cmd}{#1}%
    \minted@fvset
    \minted@pygmentize[#3]{#2}%
    \endgroup}}
\newcommand{\newminted}[3][]{
  \ifthenelse{\equal{#1}{}}
    {\def\minted@envname{#2code}}
    {\def\minted@envname{#1}}
  \newenvironment{\minted@envname}
    {\VerbatimEnvironment
      \begin{minted}[#3]{#2}}
    {\end{minted}}
  \newenvironment{\minted@envname *}[1]
    {\VerbatimEnvironment\begin{minted}[#3,##1]{#2}}
    {\end{minted}}}
\newcommand{\newmint}[3][]{
  \ifthenelse{\equal{#1}{}}
    {\def\minted@shortname{#2}}
    {\def\minted@shortname{#1}}
  \expandafter\newcommand\csname\minted@shortname\endcsname[2][]{
    \mint[#3,##1]{#2}##2}}
\newcommand{\newmintedfile}[3][]{
  \ifthenelse{\equal{#1}{}}
    {\def\minted@shortname{#2file}}
    {\def\minted@shortname{#1}}
  \expandafter\newcommand\csname\minted@shortname\endcsname[2][]{
    \inputminted[#3,##1]{#2}{##2}}}
\newcommand{\newmintinline}[3][]{%
  \ifthenelse{\equal{#1}{}}%
    {\def\minted@shortname{#2inline}}%
    {\def\minted@shortname{#1}}%
    \expandafter\newrobustcmd\csname\minted@shortname\endcsname{%
      \begingroup
      \let\do\@makeother\dospecials
      \catcode`\{=1
      \catcode`\}=2
      \@ifnextchar[{\endgroup\minted@inliner[#3][#2]}%
        {\endgroup\minted@inliner[#3][#2][]}}%
    \def\minted@inliner[##1][##2][##3]{\mintinline[##1,##3]{##2}}%
}
\ifthenelse{\boolean{minted@newfloat}}%
 {\@ifundefined{minted@float@within}%
    {\DeclareFloatingEnvironment[fileext=lol,placement=tbp]{listing}}%
    {\def\minted@tmp#1{%
       \DeclareFloatingEnvironment[fileext=lol,placement=tbp, within=#1]{listing}}%
     \expandafter\minted@tmp\expandafter{\minted@float@within}}}%
 {\@ifundefined{minted@float@within}%
    {\newfloat{listing}{tbp}{lol}}%
    {\newfloat{listing}{tbp}{lol}[\minted@float@within]}}
\ifminted@newfloat\else
\newcommand{\listingscaption}{Listing}
\floatname{listing}{\listingscaption}
\newcommand{\listoflistingscaption}{List of Listings}
\providecommand{\listoflistings}{\listof{listing}{\listoflistingscaption}}
\fi
\AtEndOfPackage{%
  \ifthenelse{\boolean{minted@draft}}%
   {}%
   {%
    \ifthenelse{\boolean{minted@frozencache}}{}{%
      \ifnum\pdf@shellescape=1\relax\else
        \PackageError{minted}%
         {You must invoke LaTeX with the
          -shell-escape flag}%
         {Pass the -shell-escape flag to LaTeX. Refer to the minted.sty
          documentation for more information.}%
      \fi}%
   }%
}
\AtEndPreamble{%
  \ifthenelse{\boolean{minted@draft}}%
   {}%
   {%
    \ifthenelse{\boolean{minted@frozencache}}{}{%
      \TestAppExists{\MintedPygmentize}%
      \ifAppExists\else
        \PackageError{minted}%
         {You must have `pygmentize' installed
          to use this package}%
         {Refer to the installation instructions in the minted
          documentation for more information.}%
      \fi}%
  }%
}
\AfterEndDocument{%
  \ifthenelse{\boolean{minted@draft}}%
   {}%
   {\ifthenelse{\boolean{minted@frozencache}}%
     {}
     {\ifx\XeTeXinterchartoks\minted@undefined
      \else
        \DeleteFile[\minted@outputdir]{\minted@jobname.mintedcmd}%
        \DeleteFile[\minted@outputdir]{\minted@jobname.mintedmd5}%
      \fi
      \DeleteFile[\minted@outputdir]{\minted@jobname.pyg}%
      \DeleteFile[\minted@outputdir]{\minted@jobname.out.pyg}%
     }%
   }%
}
\endinput
%%
%% End of file `minted.sty'.

        \end{code}