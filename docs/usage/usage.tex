%!TEX root = ../manual.tex

\section{Installation et utilisation}
    \subsection{Installation}
        Copier le répertoire \verb|texlab| n'importe où sur votre disque (la racine de votre project latex peut être une bonne idée). Dans ce manuel, ce répertoire sera copié dans \verb|C:|, et ses fichiers sont donc accessibles dans \verb|C:\texlab|.

        Pour utiliser toutes les fonctionnalités de \verb|texlab|, \href{https://www.python.org/}{python} doit être installé sur votre système, ainsi que le package \verb|pygments|, que l'on peut installer avec la commande
        \begin{code}{shell}
             pip install pygments
        \end{code} 

    \subsection{Mise en place}
        Pour utiliser \verb|texlab|, il suffit de créer un document et de commencer son préambule comme suit :
        \begin{code}{latex}
            \newcommand{\templatesroot}{C:/texlab}
            \documentclass[11pt]{article}

\usepackage[left=1.3cm,right=1.3cm,top=1.3cm,bottom=1.3cm]{geometry}

\usepackage{amsmath}
% \usepackage{amssymb}

\usepackage{dsfont}
\usepackage{bbold}
\usepackage{fontspec}
%\usepackage{etex}

\usepackage{titletoc}
\usepackage[hidelinks]{hyperref}
\usepackage[nameinlink]{cleveref}

\usepackage{tikz}

\usepackage[many]{tcolorbox}

\usepackage[table]{xcolor}
\definecolor{header_color}{gray}{0.9}
% \usepackage{bold-extra}

\usepackage{multicol}
\setlength{\columnseprule}{0.3pt}

\usepackage[french]{babel}

\usepackage{xifthen}

\usepackage{tikz}
\usetikzlibrary{positioning}

% Bullets au lieu de tirets dans les itemize
\renewcommand{\labelitemi}{\bullet}

% \usepackage{avant} % Use the Avantgarde font for headings
% \usepackage{mathptmx} % Use the Adobe Times Roman as the default text font together with math symbols from the Sym­bol, Chancery and Com­puter Modern fonts
% \usepackage{microtype} % Slightly tweak font spacing for aesthetics
% \usepackage{tgpagella}

\usepackage[bitstream-charter]{mathdesign}

% \setmainfont{Libertinus Serif}
% \setmathfont{Libertinus Math}

% \usepackage{notomath}
\usepackage[T1]{fontenc} % Use 8-bit encoding that has 256 glyphs

\usepackage{caption}

\usepackage{eurosym}

\usepackage{diagbox}
\usepackage{makecell}

% Nobreak
\newenvironment{nobreakminipage}{\noindent\begin{minipage}{\textwidth}}{\end{minipage}}
%\newenvironment{nobreakminipage}{\nopagebreak[4]}{}

\usepackage{mathtools}

\usepackage{hypcap}

\newcommand{\guillemets}[1]{\og #1 \fg{}}
\usepackage{fontspec}
%\usepackage{etex}

\usepackage{titletoc}
\usepackage[hidelinks]{hyperref}
\usepackage[nameinlink]{cleveref}

\usepackage{tikz}

\usepackage[many]{tcolorbox}

\usepackage[table]{xcolor}
\definecolor{header_color}{gray}{0.9}
% \usepackage{bold-extra}

\usepackage{multicol}
\setlength{\columnseprule}{0.3pt}

\usepackage[french]{babel}

\usepackage{xifthen}

\usepackage{tikz}
\usetikzlibrary{positioning}

% Bullets au lieu de tirets dans les itemize
\renewcommand{\labelitemi}{\bullet}

% \usepackage{avant} % Use the Avantgarde font for headings
% \usepackage{mathptmx} % Use the Adobe Times Roman as the default text font together with math symbols from the Sym­bol, Chancery and Com­puter Modern fonts
% \usepackage{microtype} % Slightly tweak font spacing for aesthetics
% \usepackage{tgpagella}

\usepackage[bitstream-charter]{mathdesign}

% \setmainfont{Libertinus Serif}
% \setmathfont{Libertinus Math}

% \usepackage{notomath}
\usepackage[T1]{fontenc} % Use 8-bit encoding that has 256 glyphs

\usepackage{caption}

\usepackage{eurosym}

\usepackage{diagbox}
\usepackage{makecell}

% Nobreak
\newenvironment{nobreakminipage}{\noindent\begin{minipage}{\textwidth}}{\end{minipage}}
%\newenvironment{nobreakminipage}{\nopagebreak[4]}{}

\usepackage{mathtools}

\usepackage{hypcap}

\newcommand{\guillemets}[1]{\og #1 \fg{}}
% \usepackage[amsmath,framed,thmmarks,hyperref]{ntheorem}
\usepackage{amsthm}

% Couleurs
\definecolor{theoremcolor}{RGB}{255,117,117}
\definecolor{definitioncolor}{RGB}{111,214,145}
\definecolor{commandcolor}{RGB}{66,173,244}
\definecolor{proofcolor}{RGB}{0,0,0}
\definecolor{exercisecolor}{RGB}{255,199,122}
\definecolor{remarkcolor}{RGB}{137,137,137}
\definecolor{methodcolor}{RGB}{105,188,155}
\definecolor{commentcolor}{RGB}{244,113,66}
\definecolor{subjectcolor}{RGB}{244,116,65}
\definecolor{testexercisecolor}{RGB}{200,200,200}
\definecolor{todolv1color}{RGB}{0,204,0}
\definecolor{todolv2color}{RGB}{255,204,0}
\definecolor{todolv3color}{RGB}{255,153,153}
\definecolor{bigtodocolor}{RGB}{180,180,180}

\newcounter{dummy}

% Général
\tcbset{
    globalbox/.style={
            enhanced jigsaw,
            boxrule=0pt,
            arc=0mm,
            outer arc=0mm,
            leftrule=4pt,
            bottomrule=0.5pt,
            toprule=0.5pt,
            rightrule=0.5pt,
            fonttitle=\bfseries,
            breakable=true,
            separator sign dash,
        }
}

% Théorèmes
\tcbset{
    theorembox/.style={
            colframe=theoremcolor,
            colback=theoremcolor!5,
            coltitle=theoremcolor!20!black,
            title style={theoremcolor!50},
        }
}

\newtcbtheorem[use counter=dummy, number within=section]{theorem}{Théorème}{globalbox,theorembox,label type=theorem}{theorems}
\crefname{theorem}{Théorème}{Théorèmes}

\newtcbtheorem[use counter=dummy, number within=section]{property}{Propriété}{globalbox,theorembox,label type=property}{properties}
\crefname{property}{Propriété}{Propriétés}

\newtcbtheorem[use counter=dummy, number within=section]{corollary}{Corollaire}{globalbox,theorembox,label type=corollary}{corollaries}
\crefname{corollary}{Corollaire}{Corollaires}

\newtcbtheorem[use counter=dummy, number within=section]{lemma}{Lemme}{globalbox,theorembox,label type=lemma}{lemmas}
\crefname{lemma}{Lemme}{Lemmes}

% Définition
\tcbset{
    definitionbox/.style={
            colframe=definitioncolor,
            colback=definitioncolor!5,
            coltitle=definitioncolor!20!black,
            title style={definitioncolor!50},
        }
}

\newtcbtheorem[use counter=dummy, number within=section]{definition}{Définition}{globalbox,definitionbox,label type=definition}{definitions}
\crefname{definition}{Définition}{Définitions}

% Commandes
\tcbset{
    commandbox/.style={
            colframe=commandcolor,
            colback=commandcolor!5,
            coltitle=commandcolor!20!black,
            title style={commandcolor!50},
        }
}

\newtcbtheorem[use counter=dummy, number within=section]{command}{Commande}{globalbox,commandbox,label type=command}{commands}
\crefname{command}{Commande}{Commandes}

% Exercices
\tcbset{
    exercisebox/.style={
            colframe=exercisecolor,
            colback=exercisecolor!5,
            coltitle=exercisecolor!20!black,
            title style={exercisecolor!50},
        }
}

\newtcbtheorem[use counter=dummy, number within=section]{exercise}{Exercice}{globalbox,exercisebox,label type=exercise}{exercises}
\crefname{exercise}{Exercice}{Exercices}

\newtcbtheorem[use counter=dummy, number within=section]{example}{Exemple}{globalbox,exercisebox,label type=example, leftrule=0pt, rightrule=0pt}{examples}
\crefname{example}{Exemple}{Exemples}

% Exercices de DS
\tcbset{
    testexercisebox/.style={
            colframe=testexercisecolor,
            colback=white,
            coltitle=testexercisecolor!20!black,
            title style={testexercisecolor!50},
        }
}

\newtcbtheorem{testexercise}{Exercice}{globalbox,testexercisebox,label type=testexercise}{testexercises}
\crefname{testexercise}{Exercice}{Exercices}

% Correction d'exercice
% \theoremheaderfont{\sc}\theorembodyfont{\upshape}
% \theoremstyle{nonumberbreak}
% \theoremseparator{}
% \theoremsymbol{\rule{1ex}{1ex}}
\theoremstyle{definition}
\newtheorem*{correction}{Correction}

% \tcolorboxenvironment
% {correction}{
%     blanker,breakable,left=5mm,
%     before skip=10pt,after skip=10pt,
%     borderline west={1mm}{0pt}{exercisecolor}}

\tcolorboxenvironment{correction}{% 'proof' from 'amsthm'
blanker,breakable,left=5mm,
before skip=10pt,after skip=10pt,
borderline west={1mm}{0pt}{exercisecolor}}

% Remarque
\tcbset{
    remarkbox/.style={
            breakable,
            enhanced,
            colback=remarkcolor!5,
            colframe=remarkcolor,
            arc=0pt,
            outer arc=0pt,
            % fonttitle=\bfseries\sffamily\large,
            fonttitle=\bfseries,
            boxrule=1pt,
            colbacktitle=remarkcolor,
            attach boxed title to top left={},
            boxed title style={
                    enhanced,
                    skin=enhancedfirst jigsaw,
                    arc=0pt,
                    bottom=0pt,
                    interior style={fill=remarkcolor}
                }
        }
}

\newtcbtheorem[use counter=dummy, number within=section]{remark}{Remarque}{remarkbox,label type=remark}{remarks}
\crefname{remark}{Remarque}{Remarques}

% Méthode
\tcbset{
    methodbox/.style={
            colback=methodcolor!5,
            colframe=methodcolor,
            colbacktitle=methodcolor,
            attach boxed title to top center={},
            boxed title style={
                    enhanced,
                    skin=enhancedfirst jigsaw,
                    arc=0pt,
                    bottom=0pt,
                    interior style={fill=methodcolor}
                }
        }
}

\newtcbtheorem[use counter=dummy, number within=section]{method}{Méthode}{remarkbox,methodbox, label type=method}{methods}
\crefname{method}{Méthode}{Méthodes}

% Commentaire
\tcbset{
    commentbox/.style={
            colback=commentcolor!5,
            colframe=commentcolor,
            colbacktitle=commentcolor,
            attach boxed title to top right={},
            boxed title style={
                    enhanced,
                    skin=enhancedfirst jigsaw,
                    arc=0pt,
                    bottom=0pt,
                    interior style={fill=commentcolor}
                }
        }
}

\newtcolorbox{note}[1][Commentaire]{remarkbox,commentbox,title=#1}

% Todos
\tcbset{
    todobox/.style={
        %     breakable,
        %     enhanced,
            arc=0pt,
            outer arc=0pt,
            fonttitle=\bfseries,
        %     boxrule=1pt,
        %     width=(\linewidth - 4pt)/3,
                width=0.33\linewidth,
            equal height group=AT,
            before=,
            after=\hfill,
        %     boxed title style={
        %             enhanced,
        %             skin=enhancedfirst jigsaw,
        %             arc=0pt,
        %             bottom=0pt,
        %         }
        }
}

\tcbset{
    bigtodobox/.style={
                enhanced,
                arc=0pt,
                outer arc=0pt,
                fonttitle=\bfseries,
            colback=white,
                % interior hidden,
            colframe=bigtodocolor,
            colbacktitle=bigtodocolor,
            left=0mm,
            right=0mm,
            top=0mm,
            bottom=0mm,
            leftrule=0pt,
            bottomrule=0.8pt,
            toprule=0.8pt,
            rightrule=0pt,
            attach boxed title to top right={},
            boxed title style={
                    enhanced,
                    skin=enhancedfirst jigsaw,
                    arc=0pt,
                    bottom=0pt,
                    interior style={fill=bigtodocolor}
                }
        }
}

\newtcolorbox{bigtodo}[1][A faire]{bigtodobox,title=#1}

\tcbset{
    todolv1box/.style={
            colback=todolv1color!5,
            colframe=todolv1color,
            colbacktitle=todolv1color,
            boxed title style={
                    interior style={fill=todolv1color}
                }
        }
}

\newtcolorbox{todolv1}[1][Niveau 1]{todobox,todolv1box,title=#1}

\tcbset{
    todolv2box/.style={
            colback=todolv2color!5,
            colframe=todolv2color,
            colbacktitle=todolv2color,
            boxed title style={
                    interior style={fill=todolv2color}
                }
        }
}

\newtcolorbox{todolv2}[1][Niveau 2]{todobox,todolv2box,title=#1}

\tcbset{
    todolv3box/.style={
            colback=todolv3color!5,
            colframe=todolv3color,
            colbacktitle=todolv3color,
            boxed title style={
                    interior style={fill=todolv3color}
                }
        }
}

\newtcolorbox{todolv3}[1][Niveau 3]{todobox,todolv3box,title=#1}

% Démonstrations
% \theoremheaderfont{\sc}\theorembodyfont{\upshape}
% \theoremstyle{nonumberbreak}
% \theoremseparator{}
% \theoremsymbol{\rule{1ex}{1ex}}
% \newtheorem{proof}{Démonstration}

% \tcolorboxenvironment
% {proof}{
%     % ’proof’ from ’amsthm’
%     blanker,breakable,left=5mm,
%     before skip=10pt,after skip=10pt,
%     borderline west={1mm}{0pt}{proofcolor}}

\tcolorboxenvironment{proof}{% 'proof' from 'amsthm'
blanker,breakable,left=5mm,
before skip=10pt,after skip=10pt,
borderline west={1mm}{0pt}{proofcolor}}

% Sujet (oral)
% \tcbset{
%     subjectbox/.style={
%             colback=subjectcolor!5,
%             colback=white,
%             colframe=subjectcolor!40,
%             colbacktitle=subjectcolor,
%             coltitle=subjectcolor!20!black,
%             % attach boxed title to top center={yshift=-2.5mm},
%             attach boxed title to top center={},
%             boxrule=1pt,
%             boxed title style={
%                     enhanced,
%                     skin=enhancedfirst jigsaw,
%                     arc=0pt,
%                     bottom=0pt,
%                     % interior style={fill=subjectcolor}
%                     interior style={fill=subjectcolor!10}
%                 }
%         }
% }

\tcbset{
    subjectbox/.style={
            colback=subjectcolor!5,
            colback=white,
            colframe=subjectcolor!40,
            colbacktitle=subjectcolor!40,
            coltitle=subjectcolor!20!black,
            center title,
            leftrule=0.5pt,
    }
}

% \newtcolorbox{kholle}[1][]{remarkbox,khollebox,breakable=true,title={Sujet de khôlle\ifthenelse{\isempty{#1}}{}{ - #1}}}
\newcounter{subject}
% \newtcbtheorem[use counter=subject]{subject}{Sujet}{remarkbox,subjectbox,breakable=true}{subjects}
\newtcbtheorem[use counter=subject]{subject}{Sujet}{globalbox,subjectbox,breakable=true}{subjects}
\newtcbtheorem[number within=subject]{subjectexercise}{Exercice}{globalbox,exercisebox,label type=exercise}{exercises}

\newtcbox{\rating}{enhanced,nobeforeafter,tcbox raise base,boxrule=0.4pt,top=0mm,bottom=0mm,
  right=0mm,left=4mm,arc=1pt,boxsep=2pt,before upper={\vphantom{dlg}},
  colframe=orange!50!black,coltext=orange!25!black,colback=orange!10!white,
  fontlower=\Large,fontupper=\scriptsize\bfseries,
  overlay={\begin{tcbclipinterior}\fill[orange!75!red!50!white] (frame.south west)
  rectangle node[text=white,font=\sffamily\bfseries\tiny,rotate=90] {PTS} ([xshift=4mm]frame.north west);\end{tcbclipinterior}}}

\newenvironment{todo}{
        \begin{bigtodo}
        \newcommand{\nextlevel}{\end{todolv1}\begin{todolv2}\renewcommand{\nextlevel}{\end{todolv2}\begin{todolv3}}}
        \begin{todolv1}
}{
        \end{todolv3}
        \end{bigtodo}
}
% !TeX root = ../manual.tex

\section{Tableaux de variation}

    Voir \href{https://zestedesavoir.com/tutoriels/439/des-tableaux-de-variations-et-de-signes-avec-latex/}{ce tutoriel}.

    \begin{latexcode}
        \begin{tikzpicture}
            \tkzTabInit[color]{$x$ / 1 , $f'(x)$ / 1, $f$ / 2} % Lignes (nom / taille)
            {$0$, $2$, $5$, $+\infty$}
            \tkzTabLine{z, -, d, h, d, +, }
            \tkzTabVar{+ / $13$, -DH / $4$,  D- / $\frac{\pi}{12}$, + / 15 }
            \tkzTabVal{3}{4}{0.5}{$\frac{\sqrt{3}}{2}$}{$7$}
        \end{tikzpicture}
    \end{latexcode}
\usepackage{minted}
\tcbuselibrary{minted}

% \newenvironment{code}[1]{\VerbatimEnvironment\begin{minted}[xleftmargin=\parindent,breaklines,linenos,frame=lines,autogobble,numbersep=5pt]{#1}}{\end{minted}}
\newcommand{\inlinecode}[2]{\mintinline{#1}|#2|}
% \usemintedstyle{manni}

\tcbset{
    codebox/.style={
        enhanced jigsaw,
        arc=1pt,
        outer arc=1pt,
        listing engine=minted,
        minted style=manni,
        minted options={breaklines,linenos,autogobble,numbersep=3mm},
        left=5mm,
        breakable,
        overlay={\begin{tcbclipinterior}\fill[red!20!blue!20!white] (frame.south west)rectangle ([xshift=5mm]frame.north west);\end{tcbclipinterior}},
        boxrule=1pt,
        colframe=gray,
    }
}

\newtcblisting{code}[1]{
    codebox,
    minted language=#1,
    listing only
}

\newtcblisting{latexcode}{
    codebox,
    minted language=latex,
    colbacklower=white
}

\newtcblisting{latexcodesbs}{
    codebox,
    minted language=latex,
    colbacklower=white,
    listing side text
}
        \end{code}