%!TEX root = ../manual.tex

\section{Théorèmes}

    Le style général des théorèmes est le suivant :

    \begin{latexcode}
        \begin{theorem}{}{}
            Voici un théorème merveilleux : $$1^2=\pth{-1}^2$$
        \end{theorem}
    \end{latexcode}

    Ils peuvent avoir des noms :

    \begin{latexcode}
        \begin{theorem}{Formule d'Euler}{}
            Pour tout réel $x$ : $$e^{ix} = \cos\pth{x} + i\sin\pth{x}$$
        \end{theorem}
    \end{latexcode}

    Ils peuvent être référencés :

    \begin{latexcode}
        \begin{property}{}{cov}
            Soient $X$ et $Y$ deux variables aléatoires discrètes, définies sur un même espace probabilisé, et admettant un moment d'ordre $2$. Si $X$ et $Y$ sont indépendantes, alors $\cov\pth{X,Y} = 0$.
        \end{property}

        La propriété \cref{properties:cov} permet de prouver que, si $X$ et $Y$ sont deux variables aléatoires discrètes définies sur un même espace probabilisé et admettant un moment d'ordre $2$, alors $X+Y$ admet une variance et $$V\pth{X+Y} = V\pth{X} + V\pth{Y}$$
    \end{latexcode}

    Voici les environnements de type théorème définis :

    \subsection{Théorèmes, propriétés, corollaires, lemmes}

    \begin{latexcode}
        \begin{theorem}{}{}
            Ceci est un théorème.
        \end{theorem}
    \end{latexcode}

    \begin{latexcode}
        \begin{property}{}{}
            Ceci est une propriété.
        \end{property}
    \end{latexcode}

    \begin{latexcode}
        \begin{lemma}{}{}
            Ceci est un lemme
        \end{lemma}
    \end{latexcode}

    \begin{latexcode}
        \begin{corollary}{}{}
            Ceci est un corollaire.
        \end{corollary}
    \end{latexcode}

    \begin{latexcode}
        \begin{proof}
            Ceci est sa démonstration.
        \end{proof}
    \end{latexcode}

    \subsection{Définitions}

    \begin{latexcode}
        \begin{definition}{}{}
            Ceci est une définition.
        \end{definition}
    \end{latexcode}
    \subsection{Commandes}

    \begin{latexcode}
        \begin{command}{}{}
            Ceci est une commande.
        \end{command}
    \end{latexcode}

    \subsection{Exercices}

    \begin{latexcode}
        \begin{exercise}{}{}
            Ceci est un exercice.
        \end{exercise}
    \end{latexcode}

    \begin{latexcode}
        \begin{correction}
            Et ceci est sa correction.
        \end{correction}
    \end{latexcode}

    \begin{latexcode}
        \begin{example}{}{}
            Ceci est un exemple.
        \end{example}
    \end{latexcode}

    \begin{latexcode}
        \begin{remark}{}{}
            Test.
        \end{remark}
    \end{latexcode}

    \begin{latexcode}
        \begin{method}{Montrer qu'une famille est libre}{}
            Test.
        \end{method}
    \end{latexcode}

    \begin{latexcode}
        \begin{note}
            Commentaire.
        \end{note}
    \end{latexcode}

    \begin{latexcode}
        \begin{note}[Note]
            Commentaire avec un titre
        \end{note}
    \end{latexcode}

    \begin{latexcode}
        \begin{subject}{}{}
            \begin{subjectexercise}{Cours}{}
                Ceci est un exercice de cours.
            \end{subjectexercise}
        \end{subject}
    \end{latexcode}
