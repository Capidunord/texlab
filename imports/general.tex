% \usepackage{fontspec}
%\usepackage{etex}

\usepackage{titletoc}
\usepackage[hidelinks]{hyperref}
% \usepackage[nameinlink]{cleveref}

\usepackage{tikz}

\usepackage[many]{tcolorbox}

\usepackage[table]{xcolor}
\definecolor{header_color}{gray}{0.9}
% \usepackage{bold-extra}

\usepackage{multicol}
\setlength{\columnseprule}{0.3pt}

\usepackage[french]{babel}\frenchsetup{StandardItemLabels=false,ItemLabeli=\textbullet,ItemLabelii=$\triangleright$,ItemLabeliii=$\diamond$}
% \renewcommand{\labelitemii}{$\triangleright$} % Change the third level bullet to a circle 

\usepackage{xifthen}

\usepackage{tikz}
\usetikzlibrary{positioning}

\usepackage[math]{cellspace}
\cellspacetoplimit 1pt
\cellspacebottomlimit 1pt

% \usepackage{avant} % Use the Avantgarde font for headings
% \usepackage{mathptmx} % Use the Adobe Times Roman as the default text font together with math symbols from the Sym­bol, Chancery and Com­puter Modern fonts
% \usepackage{microtype} % Slightly tweak font spacing for aesthetics
% \usepackage{tgpagella}

% \usepackage[bitstream-charter]{mathdesign} % Désactiver amsmath

% \usepackage{amsmath}
% \usepackage{amssymb}
% \usepackage{libertine}

\usepackage{unicode-math}
\setmainfont{XCharter}
\setmonofont{Latin Modern Mono}[Scale = MatchLowercase]
\setmathfont{XCharter Math}

% \usepackage{unicode-math}
% \setmainfont{XCharter}
% \setmonofont{Latin Modern Mono}[Scale = MatchLowercase]
% \setmathfont{XCharter Math}

% \usepackage{newcomputermodern}

% \usepackage{libertinus-otf}

% \usepackage{unicode-math}
% \setmainfont{TeX Gyre Bonum}
% \setmathfont{TeX Gyre Bonum Math}

% \usepackage{kpfonts-otf}

% \usepackage{unicode-math}
% \setmainfont{STIX Two Text}
% \setmathfont{STIX Two Math}

% \usepackage{unicode-math}
% \setmainfont{CMU Concrete}
% \setmathfont{Euler Math}

% \usepackage{fourier-otf}

% \setmainfont{Libertinus Serif}
% \setmathfont{Libertinus Math}

% \usepackage{notomath}
% \usepackage[T1]{fontenc} % Use 8-bit encoding that has 256 glyphs

\usepackage{caption}

\usepackage{eurosym}

\usepackage{diagbox}
\usepackage{makecell}

% Nobreak
% \newenvironment{nobreakminipage}{\noindent\begin{minipage}{\textwidth}}{\end{minipage}}
%\newenvironment{nobreakminipage}{\nopagebreak[4]}{}

% \usepackage{mathtools}

% \usepackage{hypcap}

\newcommand{\guillemets}[1]{\og #1 \fg{}}