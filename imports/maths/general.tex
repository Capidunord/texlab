% Parenthèses, valeurs absolues...
\newcommand{\pth}[1]{\ensuremath{\left( #1 \right)}}
\newcommand{\vabs}[1]{\ensuremath{\left| #1 \right|}}
\newcommand{\floor}[1]{\ensuremath{\left \lfloor #1 \right \rfloor}}
\newcommand{\acc}[1]{\ensuremath{\left\{ #1 \right\}}}
\newcommand{\itv}[4]{\ensuremath{
\ifthenelse{\equal{#1}{o}}{\left]}{\left[}#3;#4\ifthenelse{\equal{#2}{o}}{\right[}{\right]}
}}
\newcommand{\sqbrackets}[1]{\ensuremath{\left[ #1 \right]}}
\newcommand{\itve}[4]{\ensuremath{
\ifthenelse{\equal{#1}{o}}{\left]\!\left]}{\left[\!\left[}#3;#4\ifthenelse{\equal{#2}{o}}{\right[\!\right[}{\right]\!\right]}
}}

% Vecteurs
\newcommand{\norm}[1]{\ensuremath{\left\lVert #1 \right\rVert}}
\usepackage{esvect} % commande \vv pour les flèches sur les vecteurs (\vv{AB})

% Complexes
\newcommand{\conj}[1]{\ensuremath{\overline{#1}}}

% Limites
\newcommand{\tdsto}[1]{\ensuremath{\xrightarrow[#1]{}}}

% Probas
\newcommand{\comp}[1]{\ensuremath{\overline{#1}}}
\newcommand{\mean}[1]{\ensuremath{\overline{#1}}}

% Comparaisons (petits o, grands O, équivalents)
\newcommand{\po}[2]{\ensuremath{\underset{#1}{o}\pth{#2}}}
\newcommand{\go}[2]{\ensuremath{\underset{#1}{O}\pth{#2}}}
\newcommand{\eq}[1]{\ensuremath{\underset{#1}{\sim}}}

% Opérateurs
\DeclareMathOperator{\cov}{Cov}
\DeclareMathOperator{\vect}{Vect}
\newcommand{\dd}{\text{d}}
\DeclareMathOperator{\sh}{sh}
\DeclareMathOperator{\ch}{ch}
\DeclareMathOperator{\thh}{th}

\DeclareMathOperator{\rg}{rg}

\DeclareMathOperator{\id}{id}

\DeclareMathOperator{\mat}{Mat}

% Ensembles
\newcommand{\setN}{\mathbb{N}}
\newcommand{\setR}{\mathbb{R}}
\newcommand{\setQ}{\mathbb{Q}}
\newcommand{\setD}{\mathbb{D}}
\newcommand{\setZ}{\mathbb{Z}}
\newcommand{\setK}{\mathbb{K}}

% Calligraphie
\newcommand{\aA}{\mathcal{A}}
\newcommand{\bB}{\mathcal{B}}
\newcommand{\cC}{\mathcal{C}}
\newcommand{\mM}{\mathcal{M}}
\newcommand{\lL}{\mathcal{L}}
\newcommand{\eE}{\mathcal{E}}

% phi, epsilon
\renewcommand{\phi}{\varphi}
\renewcommand{\epsilon}{\varepsilon}

% Fonctions
\newcommand{\fonc}[5]{\ensuremath{\begin{array}{ccccl}
            #1 & \text{:} & #2 & \to     & #3 \\
               &          & #4 & \mapsto & #5
        \end{array}}}

% Transposée
\newcommand{\transp}{{}^t}